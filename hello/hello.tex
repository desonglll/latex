\documentclass[b5paper,11pt,twoside,twocolumn]{ctexbook}
% 使用fontsize宏包设置特定部分的字体大小
\usepackage{fontsize}
\usepackage{graphicx} % Required for inserting images

\usepackage{geometry}
% 设置页面边距
\geometry{
    top=2cm,
    bottom=2cm,
    left=2.5cm,
    right=2.5cm
}
\usepackage{titlesec}
% 调整章节标题的高度
\titleformat{\chapter}[hang]
  {\normalfont\LARGE\bfseries} % 标题的字体和大小
  {第\thechapter 章}{2em} % 章节编号与标题之间的距离
  {} % 章节标题之前的代码
  [\vspace{1em}] % 章节标题之后的空间

\renewcommand{\thechapter}{\chinese{chapter}}

% 使用newtxtext宏包设置默认字体为Times New Roman
\usepackage{newtxtext}
\usepackage{lipsum}

% 用於設置超連結
\usepackage{hyperref}
\hypersetup{
    colorlinks=true,
    linkcolor=black,
    filecolor=magenta,
    urlcolor=blue,
    pdftitle={Overleaf Example},
    pdfpagemode=FullScreen
}

% 用於加快編譯速度,生成PDF時註釋掉即可
% \usepackage{syntonly}
% \syntaxonly

\usepackage{tocloft}

% 配置目录的省略号
\renewcommand{\cftchapdotsep}{\cftdotsep}  % 章节标题的省略号
\renewcommand{\cftsecdotsep}{\cftdotsep}   % 节标题的省略号
\renewcommand{\cftsubsecdotsep}{\cftdotsep} % 子节标题的省略号

\title{\fontsize{18}{19}\selectfont 一把青}
\author{白先勇}
\date{April 2024}

\begin{document}

\maketitle

\clearpage

\tableofcontents

% 设置为单栏
\onecolumn
\frontmatter
\chapter{前言}

抗日胜利,还都南京的那一年,我们住在大方巷的仁爱东村,一个中下级的空军眷属区里。在四川那种闭塞的地方,煎熬了那些年数,骤然回返那六朝金粉的京都,到处的古迹,到处的繁华,一派帝王气象,把我们的眼睛都看花了。
\mainmatter
\twocolumn

\chapter{上}

抗日勝利,還都南京的那一年,我們住在大方巷的仁愛東村,一個中下級的空軍眷屬區。在四川那種閉塞的地方,煎熬了那些年數,驟然回返那六朝金粉的京都,到處的古蹟,到處的繁華,一派帝王氣象,把我們的眼睛都看花了。

那時偉成正擔任十一大隊的大隊長。他手下有兩個小隊剛從美國受訓回來,他那隊飛行員頗受重視,職務也就格外繁忙。遇到緊要差使,常由他親自率隊出馬。一個禮拜,倒有三、四天,連他的背影兒我都見不著。每次出差,他總帶著郭軒一起去。郭軒是他的得意門生,郭軒在四川灌縣航校當學生的時候,偉成就常對我說:郭軒這個小伙子靈跳過人,將來必定大有出息。果然不出幾年,郭軒便竄了上去,爬成小隊長留美去了。

郭軒是空軍的遺族。他父親是偉成的同學,老早摔了機,母親也跟著病歿了。在航校的時候,逢年過節,我總叫他到我們家吃餐團圓飯。偉成和我膝下無子,看著郭軒孤單,也常照顧他些。那時他還剃著青亮的頭皮,穿了一身土黃布的學生裝,舉止雖然處處露著聰明,可是口角到底嫩稚,還是個未經世的後生娃仔。當他從美國回來,跑到我南京的家來,衝著我倏地敬個軍禮,叫我一聲師娘時,我著實吃他唬了一跳。郭軒全身都是美式凡立丁的空軍制服,上身罩了一件翻領鑲毛的皮夾克,腰身勒得緊峭,褲帶上卻繫著一個Rav-Ban太陽眼鏡盒兒。一頂嶄新高聳的軍帽帽沿正壓在眉毛上;頭髮也蓄長了,滲黑油亮的髮腳子緊貼在兩鬢旁。才是一兩年工夫,沒料到郭軒竟出挑得英氣勃勃了。

「怎麼了,小伙子?這次回來,該有些苗頭了吧?」我笑著向他說。

「別的沒什麼,師娘,倒是在外國攢了幾百塊美金回來。」郭軒說。

「夠討老婆了!」我笑了起來。

「是呀,師娘,正在找呢。」郭軔也朝著我齜了牙齒笑道。

戰後的南京,簡直成了我們那些小飛行員的天下。無論走到哪裡,街頭巷尾,總碰到個把趾高氣揚的小空軍,手上挽了個衣著人時的小姐,瀟瀟灑灑,搖曳而過。談戀愛-個個單身的飛行員都在談戀愛。一個月我總收得到幾張偉成學生送來的結婚喜帖。可是郭軒從美國回來了年把,卻一直還沒有他的喜訊。他也帶過幾位摩登小姐到我家來吃我做的豆瓣鯉魚。事後我問起他,他總是搖搖頭笑著說:

「沒有的事,師娘,玩玩罷了。」


但有一天,他卻跑來告訴我:這次他認了真了。他愛上了一個在金陵女中念書叫朱青的女孩兒。

「師娘,」他一股勁的對我說道,「你一定會喜歡她,我要帶她去見你。師娘,我從來沒想到會對一個女孩子這樣認真過。」

郭軒那個人的性格,我倒摸得著一二。心性極為高強,年紀輕,發跡早,不免有點自負。平常談起來,他曾對我說,他必得要選中一個稱心如意的女孩兒,才肯結婚。他帶來見我的那些小姐,個個容貌不凡,他都沒有中意,我私度這個朱青大概是天仙一流的人物,才會使得郭軒如此動心。

當我見到朱青的時候,卻大大的出了意料之外。那天郭軒帶她來見我,在我家吃午餐。原來朱青卻是個十八九歲頗為單瘦的黃花閨女,來做客還穿著一身半新舊直統子的藍布長衫,襟上掖了一塊白綢子手絹兒。頭髮也沒有燙,抿得整整齊齊的垂在耳後。腳上穿了一雙帶絆的黑皮鞋,一雙白色的短統襪子倒是乾乾淨淨的。我打量了她一下,發覺她的身段還未出挑得周全,略略扁平,面皮還泛著些青白。但她的眉眼間卻蘊著一脈令人見之忘俗的水秀,見了我一頭半低著頭,靦靦腆腆,很有一股教人疼憐的怯態。一頓飯下來,我怎麼逗她,她都不大答得上腔來,一味含糊的應著。倒是郭軒在一旁卻著了忙,一忽兒替她拈菜,一忽兒替她斟茶,直慫著她跟我聊天。


「她這個人就是這麼彆扭,」郭軒到了後來急躁的指著朱青說道,「她跟我還有話說,見了人卻成了啞巴。師娘這兒又不是外人,也這麼出不得眾。」

郭軫的話語說得暴躁了些,朱青扭過頭去,羞得滿面通紅。

「算了,」我看著有點不過意,忙止住郭軒道,「朱小姐頭一次來,自然有點拘泥,你不要去戳她。吃完飯還是你們兩人去遊玄武湖去罷,那兒的荷花開得正盛呢。

郭軒是騎了他那輛十分招搖的新摩托車來的。吃完飯,他們離開的時候,郭軒把朱青扶上了後車座,幫著她繫上她那塊黑絲頭巾,然後跳上車,輕快的發動了火,向我得意洋洋的揮了揮手,倏地一下,便把朱青帶走了。朱青偎在郭軒身後,頭上那塊絲中吹得高高揚起。看著郭軒對朱青那副笑容,我知道他這次果然認了真了。

有一次,偉成回來,臉色沉得很難看,一進門便對我說:

「郭軒那小伙子越來越不像話!我倒沒料到他竟是這樣一個人」

「怎麼了?」我十分詧異,我從來沒有聽見偉成說過郭詔一句難聽的話。

「你還問得出呢!你不是知道他在追一個金陵女中的學生嗎?我看他這個人談戀愛談昏了頭!經常闖進人家學校裡去,也不管人家在上課,就去引誘那個女學生出來。我們總部來了,成個什麼體統?

郭軫被記了過,革除了小隊長的職務。當我見到郭軒時,他卻對我解說:

「師娘,不是我故意犯規,惹老師生氣,是朱青把我的心拿走了。真的,師娘,我在天上飛,我的心都在地上跟著她呢。朱青是個規規矩矩的好女孩,就是有點怕生,不大會交際罷了。定了我,現在她一個人住在一間小客棧裡還沒有著落呢。

「傻子,」我搖頭嘆道,沒想到聰明人談起戀愛來,也會變得這般糊塗,「既是這麼痴,兩人結婚算了。」

「師娘,我就是要來和你商量這件事,要請你和老師做我們的主婚人呢。」郭軒滿面光彩對我說。

郭軒和朱青結婚以後,也住在我們仁愛東村。郭軒有兩個禮拜的婚假,本來他和朱青打算到杭州去度蜜月的,可是還沒去成,猛然間國內的戰事便爆發了。偉成他們那支大隊被調到東北去。臨走的那天早上,才濛濛亮,郭軒便鑽進我的廚房裡來,我正在升火替偉成煮泡飯。郭軒披著件軍外套,頭髮蓬亂,兩眼全是紅絲,鬍鬚也沒剃,一把攥住我手,嗓子嘎啞,對我說:

「師娘,這次無論如何要拜託你老人家了——」

「曉得了,」我打斷他的話道,「你不在,自然是我來照顧你老婆啦。」

「師娘——」郭軒還在嘮叨,「朱青還不大懂事,我們空軍裡的許多規矩,她不甚明了,你要當她自己人,多多教導她才好。」

「是了,」我笑道,「你師娘跟著你老師在空軍裡混了這十來年,什麼還沒見過?不知多少人從我這裡學了乖去呢。朱青又不笨,你等我來慢慢開導她。


偉成和郭軒他們離開後,我收拾了一下屋子便走到朱青家去探望她。公家配給郭軒他們的宿舍是一幢小巧的木板平房。他們搬進去以前,郭軒特別找人粉刷過一輪,掛上些新的門簾窗幔,相當起眼。我進到他們的房子裡,看見客廳裡還是新房般的打扮。桌子椅子上堆滿了紅紅綠綠的賀禮,有些包裹尚未拆封。桌子跟下卻圍著一轉花籃,那些玫瑰劍蘭的花苞兒開得十分新鮮,連鳳尾草也是碧綠的。牆上那些喜幛也沒有收去,郭軒同學送給他的一塊烏木燙金的喜匾卻懸在廳的中央,寫著「白頭偕老」。

朱青在她房裡,我走進去她也沒聽見。她歪倒在床上,臉埋在被窩裡,抽抽搭搭的哭泣著。她身上仍舊穿著新婚的艷色絲旗袍,新燙的頭髮揉亂了,髮尾子枝椏般生硬的張著。一床繡滿五彩鴛鴦的絲被面被她搓得全是皺紋。在她臉旁被面上,卻浸著一塊碗大的濕印子。她聽見我的腳步驚坐了起來,只叫出一聲「師娘」,便只有哽咽的份兒了。朱青滿面青黃,眼睛腫得瞇了起來,看著愈加瘦弱了。我走過去替她抿了一下頭髮,絞了一把熱手巾遞給她。朱青接過手中,把臉摀住,重新又哭泣起來。房子外頭不斷的還有大卡車和吉普車在拖拉行李,鐵鍊鐵條撞擊的聲音,非常刺耳,村子裡的人正陸續啟程上任,時而女人尖叫,時而小孩啼哭,顯得十分惶亂。我等朱青哭過了,才拍拍她的肩膀說:

「頭一次,乍然分離,總是這樣的——今晚不要開夥,到我那兒吃夜飯,給我做個伴兒。」

偉成和郭軒他們一去便了無蹤跡。忽而聽見他們調到華北,忽而又來信飛到華中去了,幾個月來一次也沒回家。這個期間,朱青常常和我在一起。有時我教她做菜,有時我教她織毛衣,也有時我教她玩幾張麻將牌。

「這個玩意兒是萬靈藥,」我對她笑著說道,「有心事,坐上桌子,紅中白板一混,什麼都忘了。」

朱青結婚後,放得開多了,可是仍舊靦腆怯生,除掉我這兒,村子裡別家她一概沒有來往。村子裡那些人的身世我都知曉,漸漸兒的,我也揀了一些告訴她聽,讓她熟悉一下我們村裡那些人的生活。

「你別錯看了這些人,」我對她說,「她們背後都經過了一番歷練的呢。像你後頭那個周太太吧,她已經嫁了四次了。她現在這個丈夫和她前頭那三個原來都是一個小隊裡的人。先生原是她小叔,徐家兩兄弟都是十三大隊裡的。

「可是她們看著還有說有笑的。」朱青望著我滿面疑惑。

「我的姑娘,」我笑道,「不笑難道叫她們哭不成?要哭,也不等到現在了。」

郭軫離開後,朱青一步遠門也不肯出,天天守在村子裡。有時我們大夥兒上夫子廟去聽那些女孩們清唱,朱青也不肯跟我們去。她說她怕錯過總部打電話傳來郭軒的消息,一天夜裡,總部帶信來說,偉成那一隊經過上海,有一天多好停留,可能趕到南京來。朱青一早就跳出跳進,忙著出去買了滿滿兩籃菜回來。下午我經過她門口,看見她穿了一身藍布衣褲,頭上繫了一塊舊頭巾,站在凳子上洗窗戶。她人又矮小,踮起腳還夠不著,手裡卻揪住一塊大抹布揮來揮去,全身的勁都使出來了似的。

「朱青,那上頭的灰塵,郭軒看不見的。」我笑著叫道。

朱青回頭看見我,紅了臉,訕子的說:

「不知怎的,才幾個月,這問房子便舊了,洗也洗不乾淨。」

傍晚的時分,朱青過來邀請了我一塊兒到村口擱軍用電話的那間門房裡去等候消息。總部那邊的人答應六、七點鐘打電話給我們。朱青梳洗過了,換上一件杏黃色的薄綢長衫,頭上還絡了一根蘋果綠的絲帶,嘴上也抹了一些口紅,看著十分清新可喜。起初朱青還非常開心,跟我有說有笑,到了六點多鐘的光景,她便漸漸緊張起來了,臉也繃了,聲也噤了,她一邊織著毛線卻不時的抬頭去看桌上那架電話機。我們左等右等,直到九點多鐘,電話鈴才響了起來。朱青倏地跳起來,懷裡的絨線球滾到一地,急忙向電話奔去,可是到了桌子邊卻回過頭來向著我聲音顫抖的說道:

「師娘——電話來了。」

我去接過電話,總部裡的人說,偉成他們在上海只待了兩小時,下午五點鐘已經起飛到蘇北去了。我把這個消息告訴朱青,朱青的臉色一下子變得非常難看,她呆站著,半晌沒有出聲,臉上的肌肉卻微微的在抽搐。

「我們回去吧。」我向她說。

我們走回村子裡,朱青一直默默跟在我後面,走到我家門口時,我對她說:

「莫難過了,他們的事情很沒準的。」

朱青扭過頭去,用袖子去擂眼睛,嗓子哽咽得很厲害。

「別的沒有什麼,只是今天又空等了一天——」

我把她的肩膀摟過來說:

「朱青,師娘有幾句話想跟你講,不知你要不要聽。飛將軍的太太,不容易當。廿四小時,那顆心都掛在天上,哪怕你眼睛朝天望出血來,那天上的人未必知曉。你就得狠起心腸來,才擔得住日後的風險呢。

朱青淚眼模糊的瞅著我,似懂非懂的點著頭兒。我扳起她的下巴頦,笑著嘆道:

「回去吧,今夜早點上床。」

民國三十七年的冬天,我們這邊的戰事已經處處失利了,北邊一天天吃緊的當兒,我們東村里好幾家人都遭了兇訊。有些眷屬天天到廟裡去求神拜菩薩,算命的算命,摸骨的摸骨。我向來不信這些神鬼鬼,偉成久不來信,我便邀隔壁鄰舍來成桌牌局,熬個通宵,定定神兒。有一晚,我跟幾個鄰居正在斗牌兒,住在朱青對門的那個徐太太跑來一把將我拖了出去,上氣不接下氣的告訴我說總部剛來通知,郭軒在徐州出了事,飛機和人都跌得粉碎。我趕到朱青那兒,裡面已經黑壓擠擠滿了一屋子的人。朱青歪倒在一張靠椅上,左右一邊一個女人揪住她的膀子,把她緊緊按住,她的頭上紮了一條白毛巾,毛巾上紅殷殷的沁著巴掌大一塊血跡。我一進去,裡面的人便七嘴八舌告訴我:朱青剛才一得到消息,便抱了郭軒一套制服,往村外跑去,一邊跑一邊嚎哭,口口聲聲要去找郭軒。有人攔她,她便亂踢亂打,剛跑出村口,便一頭撞在一根鐵電線桿上,額頭上碰了一個大洞,剛才抬回來,連聲音都沒有了。

我走到朱青跟前,從別人手中接過一碗薑湯,用銅羹匙撬開朱青的牙關,紮實的灌了她幾口。她的一張臉像是劃破了的魚肚皮,一塊白,一塊紅,血汗斑斑。她的眼睛睜得老大,目光卻是散渙的。她沒有哭泣,可是兩片發青的嘴唇卻一直開合著,喉頭不斷發出一陣陣尖細的聲音,好像一隻瞎耗子被人踩得發出吱吱的慘叫來一般。我把那碗薑湯灌完了,她才漸漸的收住目光,有了幾分知覺。


朱青在床上病了許久。我把她挪到我屋子裡。日夜守住她,有時連我打牌的時候,也把她放在跟前。我怕走了眼,她又去尋短見。朱青整天睡在床上。也不說話,也不吃東西。每天都由我強灌她一點湯水。幾個禮拜,朱青便瘦得只剩下了一把骨頭,面皮死灰,眼睛凹成了兩個大窟窿。有一天我餵完她,便坐在她床邊,對她說:

「朱青,若說你是為了郭軒,你就不該這般作踐自己。就是郭軒在地下,知道了也不能心安哪。」

朱青聽了我的話,突然顫巍巍的掙扎著坐了起來,朝我點了兩下頭,冷笑道:

「他知道什麼?他跌得粉身碎骨哪裡還有知覺?他倒好,轟地一下便沒了——我也死了,可是我卻還有知覺呢。」

朱青說著,面上似哭似笑的扭曲起來,非常難看。

守了朱青個把月,自己都差不多累了。幸而她老子娘卻從重慶趕了來。她老子看見她一句話都沒有說,她娘卻狠狠的啐了一口:

「該呀!該呀!我要她莫嫁空軍,不聽話,落得這種下場!」

說著便把朱青蓬頭垢面的從床上扛下來,用板車連鋪蓋一齊拖走了。朱青才走幾天,我們也開始逃難,離開了南京。

\chapter{下}

來到台北這些年,我一直都住在長春路,我們這個眷屬區碰巧又叫做仁愛東村,可是和我在南京住的那個卻毫不相干,裡面的人四面八方遷來的都有,以前我認識的那些都不知分散到哪裡去了。幸好這些年來,日子太平,容易打發,而我們空軍裡的康樂活動,卻不輸於在南京時那麼頻繁,今天平劇。明天舞蹈,逢著節目新鮮,我也常去那些晚會去湊個熱鬧。

有一年新年,空軍新生社舉行遊藝晚會。有人說歷年來就算這次最具規模。有人送來兩張門票,我便帶了隔壁李家念中學那個女兒一同去參加。我們到了新生社的時候,晚會已經開始好一會兒了。有些人擠做一堆在搶著摸彩,可是新生廳裡卻是音樂悠揚跳舞開始了。整個新生社塞得寸步難移,男男女女,大半是年輕人,大家嘻嘻哈哈的,熱鬧得了不得。廳裡飄滿了紅紅綠綠的氣球,有幾個穿了藍色制服的小空軍,拿了煙頭燒得那些氣球砰砰嘭嘭亂炸一頓,於是一些女人便趁勢尖叫起來。夾在那些混叫混鬧的小伙子中間,我的頭都發了暈,好不容易才和李家女兒擠進了新生廳裡,我們倚在一根廳柱旁邊,觀看那些人跳舞。那晚他們弄來空軍裡一個大樂隊,總有二十來人。樂團的歌手也不少,一個個上來,衣履風流,唱了幾個流行歌,卻下到舞池和她們相識的跳舞去了。正當樂團裡那些人敲打得十分賣勁的當兒,有一個衣著分外妖燒的女人走了上來,她一站上去,底下便是一陣轟雷般的喝彩,她的風頭好像又比眾人不同一些。那個女人站在台上,笑吟吟地沒有半點兒羞態,不慌不忙把麥克風調了一下,回頭向樂隊一示意,便唱了起來。

「秦婆婆,這首歌是什麼名字?」李家女兒問道,她對流行歌還沒我在行。我的收音機,一向早上開了,睡覺才關的。

「《東山一把青》。」我答道。

這首歌,我熟得很,收音機裡常收得到白光灌的唱片,倒是難為那個女人卻也唱得出白光那股懶洋洋的浪蕩勁兒。她一手拈住麥克風,一手卻一徑滿不在乎的挑弄她那一頭蓬得像隻大鳥窩似的頭髮。她翹起下巴頦兒,一字一句,清清楚楚的唱著:

東山哪,一把青。

西山哪,一把青。

郎有心來姐有心,

郎呀,咱倆兒好成親哪——

她的身體微微傾向後面,晃過來,晃過去,然後突地一股勁兒,好像從心窩裡迸了出來似的唱道:

噯呀噯噯呀,

郎呀,咱倆兒好成親哪——

唱到過門的當兒,她便放下麥克風,走過去從一個樂師手裡拿過一雙鐵鎚般的敲打器,吱吱嚓嚓的敲打起來,一面卻在台上踏著倫巴舞步,顛倒倒,扭得頗為孟浪。她穿了一身透明紫紗灑金片的旗袍,一雙高跟鞋足有三寸高,一扭,全身的金鎖片便閃閃發光起來。一曲唱完,下面喝采聲,足有半刻時辰,於是她又隨意唱了一個才走下台來,即刻便有一群小空軍迎上去把她擁走了。我還想站著聽幾首歌,李家女兒卻吵著要到另外一個廳去摸彩去。正當我們擠出人堆離開舞池的當兒,突然有人在我身後抓住了我的膀子叫了一聲:

「師娘!」

我一回頭,看見叫我的人,赫然是剛才在台上唱「東山一把草」的那個女人。來到台北後,沒有人再叫我「師娘」了,個個都叫我秦老太,許久沒有聽到這個稱呼,驀然間,異常耳生。

「師娘,我是朱青。」那個女人笑吟吟的望著我說。

我朝她上下打量了半天,還來不及回話,一群小空軍便跑來,吵嚷著要把她挾去跳舞。她把他們摔開,湊到我耳根下說:

「你把地址給我,師娘,過兩天我接你到我家去打牌,現在我的牌張也練高了。」

她轉身時又笑吟吟的悄悄對我說:

「師娘,剛才我也是老半天才把你老人家認出來呢。」

從前看京戲,伍子胥過昭關一夜便急白了頭髮,那時我只道戲裡那樣做罷了,人的模樣兒哪裡就變得那麼厲害。那晚回家,洗臉的當兒,往鏡子裡一端詳,才猛然發覺原來自己也灑了一頭霜,難怪連朱青也認不出我來了。從前逃難的時候,只顧逃命,什麼事都懵懵懂懂的,也不知黑天白日。我們退到海南島的時候,偉成便病歿了。可笑他在天上飛了一輩子,沒有出事,坐在船上,卻硬生生的病故了。他染了痢疾,船上害病的人多,不夠藥,我看著他屙痢屙得臉發了黑。他一斷氣,船上水手便把他用麻袋套起來,和其他幾個病死的人,一齊丟到了海裡去,我只聽得「嘭」一下,人便沒了。打我嫁給偉成那天起,我心裡已經盤算好以後怎樣去收他的屍骨了。我早知道像偉成他們那種人,是活不過我的。倒是沒料到末了連他屍骨也沒收著。來到台灣,天天忙著過活,大陸上的事情,竟然逐漸淡忘了。老實說,要不是在新生社又碰見朱青,我是不會想起她來了的。

過了兩天,朱青果然差了一輛計程車帶張條子來接我去吃晚餐。原來朱青住在信義路四段,另外一個空軍眷屬區。那晚她還有其他的客人,是三個空軍小伙子,大概週未從桃園基地來台北度假的,他們也順著朱青亂叫我師娘起來,朱青指著一個白白胖胖,像個麵包似的矮子向我說:

「這是劉騷包,師娘,回頭你瞧他打牌時,那副狂骨頭的樣兒就知道了。」

那個姓劉的便湊到朱青跟前嬉皮笑臉的嚷道:

「大姐,難道今天我又撞著你什麼了?到現在還沒有半句好話呢。」

朱青只管吃吃的笑著,也不去理他,又指著另外一個瘦黑瘦黑的男人說:

「他是開小兒科醫院的,師娘只管叫他王小兒科就對了。他和我們打了這麼久的麻將,就沒和出一副體面的牌來。他是我們這裡有名的雞和大王。」

那個姓王的笑歪了嘴,說:

「大姐的話先別說絕了,回頭上了桌子,我和老劉上下手把大姐夾起來,看大姐再賭厲害。」

朱青把麵一揚,冷笑道:

「別說你們這對寶器,再換兩個厲害的來,我一樣有本事教你們輸得當了褲子才準離開這兒呢。」

朱青穿了一身布袋裝,肩上披著件紅毛衣,袖管子甩蕩甩蕩的,兩筒膀子卻露在外面。她的腰身竟變得異常豐圓起來,皮色也細緻多了,臉上畫得十分入時,本來生就一雙水盈盈的眼睛,此刻顧盼間,露出許多風情似的。接著朱青又替我介紹了一個二十來歲叫小顧的年輕男人。小顧長得比先頭那兩個體面得多,茁壯的身材,濃眉高鼻,人也厚實,不像那兩個那麼嘴滑。朱青在招呼客人的時候,小顧一徑跟在她身後,替她搬挪桌椅,聽她指揮,做些重事。

不一會,我們入了席,朱青便端上了頭一道菜來,是一盆清蒸全雞,一個琥珀色的大瓷碗裡盛著熱氣騰騰的一隻大肥母雞,朱青一放下碗,那個姓劉的便跳起來走到小顧身後,直推著他嚷道:

「小顧,快點多吃些,你們大姐燉雞來補你了。」

說著他便跟那個姓王的笑得發出了怪聲來。小顧也跟著笑了起來,臉上卻十分尷尬。朱青抓起了茶几上一頂船形軍帽,迎著姓劉的兜頭便打,姓劉的便抱了頭繞著桌子竄逃起來。那個姓王的拿起羹匙舀了一瓢雞湯送到口裡,然後舐唇咂嘴的嘆道:

「小顧來了,到底不同,大姐的雞湯都燉得下了蜜糖似的。」

朱青丟了帽子,笑得彎了腰,向那姓劉的和姓王的指點了一頓,咬著牙齒恨道:

「兩個小挨刀的,詔了大姐的雞湯,居然還吃起大姐的豆腐來!」

「大姊的豆腐自然是留給我們吃的了。」姓劉的和姓王的齊聲笑道。

「今天要不是師娘在這裡,我就要說出好話來了,」朱青走到我身邊,一隻手扶在我肩上笑著說道,「師娘,你老人家莫見怪。我原是召了這群小弟弟來侍候你老人家八圈的,哪曉得幾個小鬼頭平日被我慣壞了,嘴裡沒上沒下混說起來。

朱青用手戳了一下那個姓劉的額頭,說:

「就是你這個騷包最討人喜歡!」

說著便走進廚房裡去了。小顧也跟了進去幫朱青端菜出來。那餐飯我們吃了多久,姓劉的和姓王的便和朱青說了多久的風話。

自從那次以後,隔一兩個禮拜,朱青總要來接我到她家去一趟。可是見了她那些回數,過去的事情,她卻一句也沒提過。我們見了面總是忙著搓麻將。朱青告訴我說,小顧什麼都不愛,惟獨喜愛這幾張。他一放了假,從桃園到台北來,朱青就四處去替他兜搭子,常常連她巷子口那家雜貨店一品香老闆娘也拉了來湊腳。小顧和我們打牌的當兒,朱青便不入局,她總端張椅子,挨著小顧身後坐下,替小顧點張子。她蹺著腳,手肘子搭在小顧肩上,嘴裡卻不停的哼著歌兒,又是什麼《嘆十聲》,又是什麼《怕黃昏》,唱出各式各樣的名堂來。有時我們會打多久的牌,朱青便在旁邊哼多久的歌兒。

「你幾時學得這麼會唱歌了,朱青?」有一次我忍不住問她道,我記起她以前講話時,聲音都怕抬高些的。

「還不是剛來台灣找不到事,在空軍康樂隊裡混了這麼多年學會的。」朱青笑著答道。

「秦老太,你還不知道呀,」一品香老闆娘笑道,「我們這裡都管朱小姐叫‘賽白光’呢。」

「老闆娘又拿我來開胃了,」朱青說道,「快點用心打牌吧,回頭輸脫了底,又該你來鬧著熬通宵了。」

遇見朱青才是三、四個月的光景,有一天,我在信義路東門市場買滷味,碰見一品香的老闆娘在那兒辦貨,她一見了我就一把抓住我的膀子叫道:

「秦老太,你聽見沒有?朱小姐那個小顧上禮拜六出了事啦!他們說就在桃園的飛機場上,才起飛幾分鐘,就掉了下來。」

「我並不知道呀。」我說。

一品香老闆娘叫了一輛三輪車便和我一同往朱青家去看她去。一路上一品香老闆娘自說自話叨登了半天:

「這是怎麼說呢?好好的一個人一下子就沒了。那個小顧呀,在朱小姐家裡出入怕總有兩年多了。初時朱小姐說小顧是她幹弟弟,可是兩個人那麼眉來眼去,看著又不像。依百順,到哪裡去找? 我替朱小姐難過!

我們到了朱青家,按了半天鈴,沒有人來開門,不一會兒,卻聽見朱青隔著窗子向我們叫道:

「師娘,老闆娘,你們進來呀,門沒有閂上呢。」

我們推開門,走上她客廳裡,卻看見原來朱青正坐在窗台上,穿了一身粉紅色的綢睡衣,撈起了褲管蹺腳,在腳趾甲上塗寇丹,一頭的發捲子也沒有卸下來。她看了我們抬起頭笑道:

「我早就看見你們兩個了,指甲油沒乾,不好穿鞋子走出去開門,叫你們好等——你們來得正好,晌午我才燉了一大鍋糖醋蹄子,正愁沒人來吃。

正說著餘奶奶便走了進來。朱青慌忙從窗台上跳了下來,收了指甲油,對著一品香老闆娘說:

「老闆娘,煩你替我擺擺桌子,我進去廚房端菜來。今天都是太太們,手腳快,吃完飯起碼還有二十四圈好搓。」

朱青進去廚房,我也跟了進去幫個忙兒。茱青把鍋裡的糖醋蹄倒了出來,又架上鍋頭炒了一味豆腐。我站在她身旁端著盤子等著替她盛菜。

「小顧出了事,師娘該聽到了?」朱青一邊炒菜,頭也沒有回,便對我說道。

「剛才一品香老闆娘告訴我了。」我說。

「小顧這裡沒有親人。他的後事由我和他幾個同學料理清楚了。昨天下午,我才把他的骨灰運到碧潭公墓下了葬。」

我站在朱青身後,瞅著她,沒有說話,朱青臉上沒有施脂粉,可是看著還是異樣的年輕朗爽,全不像個三十來歲的婦人,大概她的雙頰豐腴了,肌膚也緊滑了,歲月在她的臉上好像刻不下痕跡來了似的。我覺得雖然我比朱青還大了一大把年紀,可是我已經找不出什麼話來可以開導她的了。朱青俐落的把豆腐兩翻便起了鍋,然後舀了一瓢,送到我嘴裡,笑著說:

「師娘嚐嚐我的‘麻婆豆婆’,可夠味了沒有?」

我們吃過飯,朱青便擺下麻將桌子,把她待客用的那副蘇州竹子牌拿了出來。我們一坐下去,頭一盤,朱青便撂下一副大三元來。

「朱小姐,」一品香老闆娘嚷道,「你的運氣這樣好,該去買‘愛國獎券’了!」

「你們且試著吧,」朱青笑道,「今天我的風頭又要來了。」

八圈上頭,便成了三歸一的局面,朱青面前的籌碼堆到鼻尖上去了。朱青不停的笑聲,嘴裡翻來滾去哼著她常愛唱的那首《東山一把青》。隔不了一會兒,她便哼出兩句:

噯呀噯噯呀,

郎呀,採花兒要趁早哪——

% 设置目录中附录标题为繁体中文
\renewcommand{\appendixname}{附錄}
\appendix
\chapter{這是附錄A}

\backmatter
This is \textbackslash backmatter

\end{document}
