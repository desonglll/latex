% !TEX program = XeLaTeX
% !TEX encoding = UTF-8
\documentclass[a4paper,11pt,twoside,UTF8,fontset=none]{ctexart}
\input{styles/font.sty}
% 使用fontsize宏包设置特定部分的字体大小
\usepackage{fontsize}
\usepackage{graphicx} % Required for inserting images
\usepackage{geometry}
% 设置页面边距
\geometry{
    top=2cm,
    bottom=2cm,
    inner=1.5cm,
    outer=2cm,
    bindingoffset=1cm
}
% 使用newtxtext宏包设置默认字体为Times New Roman
\usepackage{newtxtext}
\usepackage{lipsum}
% 用於設置超連結
\usepackage{hyperref}
\hypersetup{
    colorlinks=true,
    linkcolor=black,
    filecolor=magenta,
    urlcolor=blue,
    pdftitle={Overleaf Example},
    pdfpagemode=FullScreen
}
% 用於加快編譯速度,生成PDF時註釋掉即可
% \usepackage{syntonly}
% \syntaxonly
\usepackage{tocloft}



% Custom packages
\usepackage{styles/hdr_ftr}
\usepackage{styles/linebreak}
% 配置目录的省略号
\renewcommand{\cftsecdotsep}{\cftdotsep}   % 节标题的省略号
\renewcommand{\cftsubsecdotsep}{\cftdotsep} % 子节标题的省略号

% 将section编号设置为中文数字
\renewcommand{\thesection}{第\chinese{section}節}

\newcommand{\mytitle}{智能交互实验教学公司}
\newcommand{\myauthor}{}
\newcommand{\mydate}{\today}

\begin{document}
\setlength{\headheight}{14pt}
% 创建自定义标题页
\begin{titlepage}
    \centering
    % 标题
    {\Huge \bfseries \mytitle\par}
    \vspace{1cm}

    % 插入图片
    \includegraphics[width=0.8\textwidth]{cover.jpg}\par
    \vspace{1cm}

    % 作者
    {\Large \myauthor\par}
    \vspace{0.5cm}

    % 日期
    {\Large \mydate\par}
    \vspace{2cm}

\end{titlepage}

\onecolumn

\pagestyle{empty}

\cleardoublepage

\tableofcontents

\cleardoublepage

\section*{一把青~維基簡介}
\addcontentsline{toc}{section}{一把青~維基簡介}
《一把青》(英語:A Touch of Green),2015年公視國語旗艦影集、時代劇。本劇前期製作費時3年,製作團隊步履遍及臺灣、南京、上海並進行勘景、選角和考據。由導演曹瑞原改編自白先勇同名短篇小說作品。中華民國文化部2013年(民國102年)高畫質電視節目旗艦型連續劇類補助新臺幣6000萬元,總製作預算高達1億8000萬元,2015年3月10日開拍,10月20日殺青。公視於12月19日上檔,並與LINE TV同步播出。LiTV 線上影視於2023年8月16日全劇上架。由楊謹華、天心、楊一展、藍鈞天、吳慷仁、連俞涵、鍾承翰、温貞菱領銜主演。

本劇描寫1945年到1981年間,從中華民國在第二次中日戰爭抗戰勝利後到第二次國共內戰再到後來中華民國政府遷臺,那個「消滅萬惡共匪、解救大陸同胞」及「漢賊不兩立」等反攻大陸口號喊得震天價響的動盪不安年代,空軍飛官及其眷屬幾段被戰火無情摧逼,生死兩隔的故事。

導演曹瑞原以劇中的朱青、秦芊儀、周瑋訓(小周)三個空軍女眷,從青澀的女學生嫁給空軍飛官後,跟著丈夫從南京來到台灣的眷村,經歷遷徙、漂泊與喪夫之慟的離散。在那個顛沛流離的大時代中,雖遭遇苦難,卻彼此因著「人性之愛」相互幫補扶持,帶出女性總能扛起一個時代的堅韌動人故事。

故事背景之史實包含:

1945-1949年(第1-20集):八年抗戰、戡亂戰爭(1946年南京學潮以及東北戰役為主)、1949年中華民國政府遷台
-1981年(第21-31集):兩岸武力對峙、臺灣白色恐怖時期、黑蝙蝠中隊、黑貓中隊。

\cleardoublepage

\twocolumn

\pagestyle{fancy}

\section*{角色簡介}
\addcontentsline{toc}{section}{角色簡介} % 加入到目錄
% 定義新的 \customsubsection 命令
\newcommand{\csubsection}[1]{
    % 增加 subsection 计数器值 1
    % \addtocounter{subsection}{1}
    \subsection*{#1} % 取消編號的標題
    \addcontentsline{toc}{subsection}{#1} % 加入到目錄
}
\csubsection{秦芊儀}

人稱師娘,華南師範大學英文科退學,家族為浙江望族,抗戰前夕,嫁給十一大隊江偉成而放棄學業。外柔內剛的師娘,一路陪伴著偉成,並打理著空軍村裏的一切。她的青春,化作成跑道盡頭的行燈,在大時代裡,微弱地閃爍著……她展現了那個年代女人的樣貌,溫柔卻剛強,為了愛情,不顧家人反對堅持嫁給偉成,從此過著平凡、但沒有安全感的生活,卻能以智慧和堅韌態度面對一切,一路照顧與開導,帶領女人們繼續向前,曾經一度被認定是匪諜而入獄,後來被釋放,最後留在空軍村過完一生。

\csubsection{江偉成}

空軍中校大隊長,飛行技術精湛,性格剛毅果斷並嚴賞罰。從抗戰勝利、眾所景仰的英雄時的意氣風發,到為了陪伴妻子,考慮轉參謀職的內心交戰,至內戰爆發,指揮作戰,失去大隊所有飛機和飛行員之後,內心自責與後半生的痛苦,而後輾轉到台灣後過著清貧生活。晚景的淒涼與無奈,是整個動盪時代下,諸多戰爭老兵的象徵人物,最後為了不要拖累芊儀上吊自殺。原著中在海上染病死亡,沒有跟隨秦芊儀來到台灣。


\csubsection{周瑋訓}

小周,空軍遺屬軍眷,人稱副隊娘。父為前清武舉人,家族為遼北省望族。抗戰時期就讀於華南師範大學體育科,在校時與師娘結為手帕交。其夫原為第十一大隊副隊長,對日抗戰時殉職,之後依著空軍傳統,學長陣亡,托妻學弟,改嫁小邵。性格直率的她,常用她的大嗓門,嬉笑怒罵的掩飾內心荒寂。離亂歲月,每在時代嘯浪裡安然漂出的她,祇能任著師娘及朱青漸漸偏離了人生航線,無能為力……小周的故事最能代表空軍特殊「傳統」的角色,面對隨時可能發生的死亡與別離,在堅強中掩藏著脆弱。她的個性率真,情感卻細膩,活潑中帶點潑辣的個性,讓有她在的場合總能帶來溫暖與歡笑。口頭禪:「狗肉進不了大上海」,後來搬離空軍村。原著中僅透過芊儀的話出場。

\newpage

\csubsection{邵志堅}

小邵,第十一大隊少校副隊長,到台灣後為十一大隊中校大隊長。個性正直、謹慎,遇事總瞻前顧後,也因如此,總在需要果斷抉擇時錯失良機。最初在南京原任大隊上尉作戰官,因原任副隊長代其迎戰日軍身亡,情義使然,放棄了與女友的感情,依空軍傳統和同袍遺願,「交接」小周與墨婷母女,自此生活在責任與摯愛的兩難之中,最後在台灣因匪諜案牽連,從十一大隊中校大隊長降調空軍工程隊少校隊長。


\csubsection{郭軫}

原就讀西北大學工程系航空工程組,對日抗戰時慨然投筆從戎,考入中央航校,第十一大隊中尉分隊長,是偉成的得意門生,對同袍有情有義的他,也因空中作戰的命懸一線,笑看人生,玩世不恭。作戰負傷,於浙江聯中女生宿舍養傷之際,留下祇有編號未有姓名的字條。未料青春少艾的朱青竟依字條千里尋來。
漂泊靈魂終有自己的導航塔,他與朱青產生一段轟轟烈烈、刻骨銘心,但卻短暫的愛情。


\csubsection{朱青}

一個青澀杭州師範女學生,帶著少女對愛情的渴望,因為一張空軍飛行員郭軫留下的字條來到南京。原本,她祇好奇那留字條的飛行員長什麼樣子,但千里見得郭軫後,她中斷了金陵女子大學的學業,與郭軫相戀,嫁進空軍眷村。然而最後等待她的,只有失去座標的人生。樸素、純潔的女學生,經歷了慘痛的生離死別而後變得成熟世故,判若兩人。心死之後,卻以玩世態度面對人生,麻木不仁,再也沒有什麼事可以傷害她,曾經一度被認定是匪諜而入獄,後來跟芊儀互換而出獄,因顧肇鈞幫助而實現了到美國定居的願望,最後回到台灣跟墨婷在前空軍機棚重逢。


\csubsection{顧肇鈞}

小顧,菜鳥飛官,抗戰末期考入航校。畢業後,對日抗戰已結束,往各大隊任准尉見習官見習。均被考評為不適飛行。小顧個性執拗,堅持要當飛行員。上級給了最後機會-調入十一大隊作最後考評。來到十一大隊後遇見已嫁給郭軫的朱青。此時小顧發現繫起朱青、郭軫千里姻緣的字條,自己才是原來的作者,自以為自己才是字條主人,甘冒大不韙,執著於對學長妻子朱青的愛戀。到台灣之後再見朱青,對歷經滄桑的她更加憐惜,以自己的方式表達對朱青的愛,在最後任務中失蹤(疑似被擊落)。


\csubsection{靳墨婷(邵墨婷)}

編劇黃世鳴以旁觀者的手法,讓墨婷時而成為劇中人,又時而成為描述劇情的旁白者。她聰慧早熟,年紀雖小,卻看盡了那段離亂的時代-在母親及兩個乾媽的身邊,親眼目睹那場戰爭的震盪,對男人、對女人、對一切生命的,無情催逼。生父曾是飛行大隊副隊長,在抗戰時期遭日軍擊落而殉職,後因母親小周改嫁,成為一位擁有兩位父親姓氏的女孩。之後隨部隊撤退到台灣,展開了她自己的青春,和焦飛開展一段屬於眷村第二代的生命故事,也看盡了空軍村裡的悲歡離合。

\include{sections/part_A}


\section{下部}

\sloppy

來到台北這些年,我一直都住在長春路,我們這個眷屬區碰巧又叫做仁愛東村,可是和我在南京住的那個卻毫不相干,裡面的人四面八方遷來的都有,以前我認識的那些都不知分散到哪裡去了。幸好這些年來,日子太平,容易打發,而我們空軍裡的康樂活動,卻不輸於在南京時那麼頻繁,今天平劇。明天舞蹈,逢著節目新鮮,我也常去那些晚會去湊個熱鬧。

有一年新年,空軍新生社舉行遊藝晚會。有人說歷年來就算這次最具規模。有人送來兩張門票,我便帶了隔壁李家念中學那個女兒一同去參加。我們到了新生社的時候,晚會已經開始好一會兒了。有些人擠做一堆在搶著摸彩,可是新生廳裡卻是音樂悠揚跳舞開始了。整個新生社塞得寸步難移,男男女女,大半是年輕人,大家嘻嘻哈哈的,熱鬧得了不得。廳裡飄滿了紅紅綠綠的氣球,有幾個穿了藍色制服的小空軍,拿了煙頭燒得那些氣球砰砰嘭嘭亂炸一頓,於是一些女人便趁勢尖叫起來。夾在那些混叫混鬧的小伙子中間,我的頭都發了暈,好不容易才和李家女兒擠進了新生廳裡,我們倚在一根廳柱旁邊,觀看那些人跳舞。那晚他們弄來空軍裡一個大樂隊,總有二十來人。樂團的歌手也不少,一個個上來,衣履風流,唱了幾個流行歌,卻下到舞池和她們相識的跳舞去了。正當樂團裡那些人敲打得十分賣勁的當兒,有一個衣著分外妖燒的女人走了上來,她一站上去,底下便是一陣轟雷般的喝彩,她的風頭好像又比眾人不同一些。那個女人站在台上,笑吟吟地沒有半點兒羞態,不慌不忙把麥克風調了一下,回頭向樂隊一示意,便唱了起來。

「秦婆婆,這首歌是什麼名字?」李家女兒問道,她對流行歌還沒我在行。我的收音機,一向早上開了,睡覺才關的。

「《東山一把青》。」我答道。

這首歌,我熟得很,收音機裡常收得到白光灌的唱片,倒是難為那個女人卻也唱得出白光那股懶洋洋的浪蕩勁兒。她一手拈住麥克風,一手卻一徑滿不在乎的挑弄她那一頭蓬得像隻大鳥窩似的頭髮。她翹起下巴頦兒,一字一句,清清楚楚的唱著:

東山哪,一把青。

西山哪,一把青。

郎有心來姐有心,

郎呀,咱倆兒好成親哪——

她的身體微微傾向後面,晃過來,晃過去,然後突地一股勁兒,好像從心窩裡迸了出來似的唱道:

噯呀噯噯呀,

郎呀,咱倆兒好成親哪——

唱到過門的當兒,她便放下麥克風,走過去從一個樂師手裡拿過一雙鐵鎚般的敲打器,吱吱嚓嚓的敲打起來,一面卻在台上踏著倫巴舞步,顛倒倒,扭得頗為孟浪。她穿了一身透明紫紗灑金片的旗袍,一雙高跟鞋足有三寸高,一扭,全身的金鎖片便閃閃發光起來。一曲唱完,下面喝采聲,足有半刻時辰,於是她又隨意唱了一個才走下台來,即刻便有一群小空軍迎上去把她擁走了。我還想站著聽幾首歌,李家女兒卻吵著要到另外一個廳去摸彩去。正當我們擠出人堆離開舞池的當兒,突然有人在我身後抓住了我的膀子叫了一聲:

「師娘!」

我一回頭,看見叫我的人,赫然是剛才在台上唱「東山一把草」的那個女人。來到台北後,沒有人再叫我「師娘」了,個個都叫我秦老太,許久沒有聽到這個稱呼,驀然間,異常耳生。

「師娘,我是朱青。」那個女人笑吟吟的望著我說。

我朝她上下打量了半天,還來不及回話,一群小空軍便跑來,吵嚷著要把她挾去跳舞。她把他們摔開,湊到我耳根下說:

「你把地址給我,師娘,過兩天我接你到我家去打牌,現在我的牌張也練高了。」

她轉身時又笑吟吟的悄悄對我說:

「師娘,剛才我也是老半天才把你老人家認出來呢。」

從前看京戲,伍子胥過昭關一夜便急白了頭髮,那時我只道戲裡那樣做罷了,人的模樣兒哪裡就變得那麼厲害。那晚回家,洗臉的當兒,往鏡子裡一端詳,才猛然發覺原來自己也灑了一頭霜,難怪連朱青也認不出我來了。從前逃難的時候,只顧逃命,什麼事都懵懵懂懂的,也不知黑天白日。我們退到海南島的時候,偉成便病歿了。可笑他在天上飛了一輩子,沒有出事,坐在船上,卻硬生生的病故了。他染了痢疾,船上害病的人多,不夠藥,我看著他屙痢屙得臉發了黑。他一斷氣,船上水手便把他用麻袋套起來,和其他幾個病死的人,一齊丟到了海裡去,我只聽得「嘭」一下,人便沒了。打我嫁給偉成那天起,我心裡已經盤算好以後怎樣去收他的屍骨了。我早知道像偉成他們那種人,是活不過我的。倒是沒料到末了連他屍骨也沒收著。來到台灣,天天忙著過活,大陸上的事情,竟然逐漸淡忘了。老實說,要不是在新生社又碰見朱青,我是不會想起她來了的。

過了兩天,朱青果然差了一輛計程車帶張條子來接我去吃晚餐。原來朱青住在信義路四段,另外一個空軍眷屬區。那晚她還有其他的客人,是三個空軍小伙子,大概週未從桃園基地來台北度假的,他們也順著朱青亂叫我師娘起來,朱青指著一個白白胖胖,像個麵包似的矮子向我說:

「這是劉騷包,師娘,回頭你瞧他打牌時,那副狂骨頭的樣兒就知道了。」

那個姓劉的便湊到朱青跟前嬉皮笑臉的嚷道:

「大姐,難道今天我又撞著你什麼了?到現在還沒有半句好話呢。」

朱青只管吃吃的笑著,也不去理他,又指著另外一個瘦黑瘦黑的男人說:

「他是開小兒科醫院的,師娘只管叫他王小兒科就對了。他和我們打了這麼久的麻將,就沒和出一副體面的牌來。他是我們這裡有名的雞和大王。」

那個姓王的笑歪了嘴,說:

「大姐的話先別說絕了,回頭上了桌子,我和老劉上下手把大姐夾起來,看大姐再賭厲害。」

朱青把麵一揚,冷笑道:

「別說你們這對寶器,再換兩個厲害的來,我一樣有本事教你們輸得當了褲子才準離開這兒呢。」

朱青穿了一身布袋裝,肩上披著件紅毛衣,袖管子甩蕩甩蕩的,兩筒膀子卻露在外面。她的腰身竟變得異常豐圓起來,皮色也細緻多了,臉上畫得十分入時,本來生就一雙水盈盈的眼睛,此刻顧盼間,露出許多風情似的。接著朱青又替我介紹了一個二十來歲叫小顧的年輕男人。小顧長得比先頭那兩個體面得多,茁壯的身材,濃眉高鼻,人也厚實,不像那兩個那麼嘴滑。朱青在招呼客人的時候,小顧一徑跟在她身後,替她搬挪桌椅,聽她指揮,做些重事。

不一會,我們入了席,朱青便端上了頭一道菜來,是一盆清蒸全雞,一個琥珀色的大瓷碗裡盛著熱氣騰騰的一隻大肥母雞,朱青一放下碗,那個姓劉的便跳起來走到小顧身後,直推著他嚷道:

「小顧,快點多吃些,你們大姐燉雞來補你了。」

說著他便跟那個姓王的笑得發出了怪聲來。小顧也跟著笑了起來,臉上卻十分尷尬。朱青抓起了茶几上一頂船形軍帽,迎著姓劉的兜頭便打,姓劉的便抱了頭繞著桌子竄逃起來。那個姓王的拿起羹匙舀了一瓢雞湯送到口裡,然後舐唇咂嘴的嘆道:

「小顧來了,到底不同,大姐的雞湯都燉得下了蜜糖似的。」

朱青丟了帽子,笑得彎了腰,向那姓劉的和姓王的指點了一頓,咬著牙齒恨道:

「兩個小挨刀的,詔了大姐的雞湯,居然還吃起大姐的豆腐來!」

「大姊的豆腐自然是留給我們吃的了。」姓劉的和姓王的齊聲笑道。

「今天要不是師娘在這裡,我就要說出好話來了,」朱青走到我身邊,一隻手扶在我肩上笑著說道,「師娘,你老人家莫見怪。我原是召了這群小弟弟來侍候你老人家八圈的,哪曉得幾個小鬼頭平日被我慣壞了,嘴裡沒上沒下混說起來。

朱青用手戳了一下那個姓劉的額頭,說:

「就是你這個騷包最討人喜歡!」

說著便走進廚房裡去了。小顧也跟了進去幫朱青端菜出來。那餐飯我們吃了多久,姓劉的和姓王的便和朱青說了多久的風話。

自從那次以後,隔一兩個禮拜,朱青總要來接我到她家去一趟。可是見了她那些回數,過去的事情,她卻一句也沒提過。我們見了面總是忙著搓麻將。朱青告訴我說,小顧什麼都不愛,惟獨喜愛這幾張。他一放了假,從桃園到台北來,朱青就四處去替他兜搭子,常常連她巷子口那家雜貨店一品香老闆娘也拉了來湊腳。小顧和我們打牌的當兒,朱青便不入局,她總端張椅子,挨著小顧身後坐下,替小顧點張子。她蹺著腳,手肘子搭在小顧肩上,嘴裡卻不停的哼著歌兒,又是什麼《嘆十聲》,又是什麼《怕黃昏》,唱出各式各樣的名堂來。有時我們會打多久的牌,朱青便在旁邊哼多久的歌兒。

「你幾時學得這麼會唱歌了,朱青?」有一次我忍不住問她道,我記起她以前講話時,聲音都怕抬高些的。

「還不是剛來台灣找不到事,在空軍康樂隊裡混了這麼多年學會的。」朱青笑著答道。

「秦老太,你還不知道呀,」一品香老闆娘笑道,「我們這裡都管朱小姐叫‘賽白光’呢。」

「老闆娘又拿我來開胃了,」朱青說道,「快點用心打牌吧,回頭輸脫了底,又該你來鬧著熬通宵了。」

遇見朱青才是三、四個月的光景,有一天,我在信義路東門市場買滷味,碰見一品香的老闆娘在那兒辦貨,她一見了我就一把抓住我的膀子叫道:

「秦老太,你聽見沒有?朱小姐那個小顧上禮拜六出了事啦!他們說就在桃園的飛機場上,才起飛幾分鐘,就掉了下來。」

「我並不知道呀。」我說。

一品香老闆娘叫了一輛三輪車便和我一同往朱青家去看她去。一路上一品香老闆娘自說自話叨登了半天:

「這是怎麼說呢?好好的一個人一下子就沒了。那個小顧呀,在朱小姐家裡出入怕總有兩年多了。初時朱小姐說小顧是她幹弟弟,可是兩個人那麼眉來眼去,看著又不像。依百順,到哪裡去找? 我替朱小姐難過!

我們到了朱青家,按了半天鈴,沒有人來開門,不一會兒,卻聽見朱青隔著窗子向我們叫道:

「師娘,老闆娘,你們進來呀,門沒有閂上呢。」

我們推開門,走上她客廳裡,卻看見原來朱青正坐在窗台上,穿了一身粉紅色的綢睡衣,撈起了褲管蹺腳,在腳趾甲上塗寇丹,一頭的發捲子也沒有卸下來。她看了我們抬起頭笑道:

「我早就看見你們兩個了,指甲油沒乾,不好穿鞋子走出去開門,叫你們好等——你們來得正好,晌午我才燉了一大鍋糖醋蹄子,正愁沒人來吃。

正說著餘奶奶便走了進來。朱青慌忙從窗台上跳了下來,收了指甲油,對著一品香老闆娘說:

「老闆娘,煩你替我擺擺桌子,我進去廚房端菜來。今天都是太太們,手腳快,吃完飯起碼還有二十四圈好搓。」

朱青進去廚房,我也跟了進去幫個忙兒。茱青把鍋裡的糖醋蹄倒了出來,又架上鍋頭炒了一味豆腐。我站在她身旁端著盤子等著替她盛菜。

「小顧出了事,師娘該聽到了?」朱青一邊炒菜,頭也沒有回,便對我說道。

「剛才一品香老闆娘告訴我了。」我說。

「小顧這裡沒有親人。他的後事由我和他幾個同學料理清楚了。昨天下午,我才把他的骨灰運到碧潭公墓下了葬。」

我站在朱青身後,瞅著她,沒有說話,朱青臉上沒有施脂粉,可是看著還是異樣的年輕朗爽,全不像個三十來歲的婦人,大概她的雙頰豐腴了,肌膚也緊滑了,歲月在她的臉上好像刻不下痕跡來了似的。我覺得雖然我比朱青還大了一大把年紀,可是我已經找不出什麼話來可以開導她的了。朱青俐落的把豆腐兩翻便起了鍋,然後舀了一瓢,送到我嘴裡,笑著說:

「師娘嚐嚐我的‘麻婆豆婆’,可夠味了沒有?」

我們吃過飯,朱青便擺下麻將桌子,把她待客用的那副蘇州竹子牌拿了出來。我們一坐下去,頭一盤,朱青便撂下一副大三元來。

「朱小姐,」一品香老闆娘嚷道,「你的運氣這樣好,該去買‘愛國獎券’了!」

「你們且試著吧,」朱青笑道,「今天我的風頭又要來了。」

八圈上頭,便成了三歸一的局面,朱青面前的籌碼堆到鼻尖上去了。朱青不停的笑聲,嘴裡翻來滾去哼著她常愛唱的那首《東山一把青》。隔不了一會兒,她便哼出兩句:

噯呀噯噯呀,

郎呀,採花兒要趁早哪——

% % 设置目录中附录标题为繁体中文
% \renewcommand{\appendixname}{附錄}
% \appendix
% \chapter{這是附錄A}

% \backmatter
% This is \textbackslash backmatter

\fussy

\begin{verbatim}
    pub fn main() {
        println!("hello world!");
    }
\end{verbatim}

\texttt{print}

\end{document}
