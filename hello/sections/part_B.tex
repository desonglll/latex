
\section{下部}

\sloppy

來到台北這些年,我一直都住在長春路,我們這個眷屬區碰巧又叫做仁愛東村,可是和我在南京住的那個卻毫不相干,裡面的人四面八方遷來的都有,以前我認識的那些都不知分散到哪裡去了。幸好這些年來,日子太平,容易打發,而我們空軍裡的康樂活動,卻不輸於在南京時那麼頻繁,今天平劇。明天舞蹈,逢著節目新鮮,我也常去那些晚會去湊個熱鬧。

有一年新年,空軍新生社舉行遊藝晚會。有人說歷年來就算這次最具規模。有人送來兩張門票,我便帶了隔壁李家念中學那個女兒一同去參加。我們到了新生社的時候,晚會已經開始好一會兒了。有些人擠做一堆在搶著摸彩,可是新生廳裡卻是音樂悠揚跳舞開始了。整個新生社塞得寸步難移,男男女女,大半是年輕人,大家嘻嘻哈哈的,熱鬧得了不得。廳裡飄滿了紅紅綠綠的氣球,有幾個穿了藍色制服的小空軍,拿了煙頭燒得那些氣球砰砰嘭嘭亂炸一頓,於是一些女人便趁勢尖叫起來。夾在那些混叫混鬧的小伙子中間,我的頭都發了暈,好不容易才和李家女兒擠進了新生廳裡,我們倚在一根廳柱旁邊,觀看那些人跳舞。那晚他們弄來空軍裡一個大樂隊,總有二十來人。樂團的歌手也不少,一個個上來,衣履風流,唱了幾個流行歌,卻下到舞池和她們相識的跳舞去了。正當樂團裡那些人敲打得十分賣勁的當兒,有一個衣著分外妖燒的女人走了上來,她一站上去,底下便是一陣轟雷般的喝彩,她的風頭好像又比眾人不同一些。那個女人站在台上,笑吟吟地沒有半點兒羞態,不慌不忙把麥克風調了一下,回頭向樂隊一示意,便唱了起來。

「秦婆婆,這首歌是什麼名字?」李家女兒問道,她對流行歌還沒我在行。我的收音機,一向早上開了,睡覺才關的。

「《東山一把青》。」我答道。

這首歌,我熟得很,收音機裡常收得到白光灌的唱片,倒是難為那個女人卻也唱得出白光那股懶洋洋的浪蕩勁兒。她一手拈住麥克風,一手卻一徑滿不在乎的挑弄她那一頭蓬得像隻大鳥窩似的頭髮。她翹起下巴頦兒,一字一句,清清楚楚的唱著:

東山哪,一把青。

西山哪,一把青。

郎有心來姐有心,

郎呀,咱倆兒好成親哪——

她的身體微微傾向後面,晃過來,晃過去,然後突地一股勁兒,好像從心窩裡迸了出來似的唱道:

噯呀噯噯呀,

郎呀,咱倆兒好成親哪——

唱到過門的當兒,她便放下麥克風,走過去從一個樂師手裡拿過一雙鐵鎚般的敲打器,吱吱嚓嚓的敲打起來,一面卻在台上踏著倫巴舞步,顛倒倒,扭得頗為孟浪。她穿了一身透明紫紗灑金片的旗袍,一雙高跟鞋足有三寸高,一扭,全身的金鎖片便閃閃發光起來。一曲唱完,下面喝采聲,足有半刻時辰,於是她又隨意唱了一個才走下台來,即刻便有一群小空軍迎上去把她擁走了。我還想站著聽幾首歌,李家女兒卻吵著要到另外一個廳去摸彩去。正當我們擠出人堆離開舞池的當兒,突然有人在我身後抓住了我的膀子叫了一聲:

「師娘!」

我一回頭,看見叫我的人,赫然是剛才在台上唱「東山一把草」的那個女人。來到台北後,沒有人再叫我「師娘」了,個個都叫我秦老太,許久沒有聽到這個稱呼,驀然間,異常耳生。

「師娘,我是朱青。」那個女人笑吟吟的望著我說。

我朝她上下打量了半天,還來不及回話,一群小空軍便跑來,吵嚷著要把她挾去跳舞。她把他們摔開,湊到我耳根下說:

「你把地址給我,師娘,過兩天我接你到我家去打牌,現在我的牌張也練高了。」

她轉身時又笑吟吟的悄悄對我說:

「師娘,剛才我也是老半天才把你老人家認出來呢。」

從前看京戲,伍子胥過昭關一夜便急白了頭髮,那時我只道戲裡那樣做罷了,人的模樣兒哪裡就變得那麼厲害。那晚回家,洗臉的當兒,往鏡子裡一端詳,才猛然發覺原來自己也灑了一頭霜,難怪連朱青也認不出我來了。從前逃難的時候,只顧逃命,什麼事都懵懵懂懂的,也不知黑天白日。我們退到海南島的時候,偉成便病歿了。可笑他在天上飛了一輩子,沒有出事,坐在船上,卻硬生生的病故了。他染了痢疾,船上害病的人多,不夠藥,我看著他屙痢屙得臉發了黑。他一斷氣,船上水手便把他用麻袋套起來,和其他幾個病死的人,一齊丟到了海裡去,我只聽得「嘭」一下,人便沒了。打我嫁給偉成那天起,我心裡已經盤算好以後怎樣去收他的屍骨了。我早知道像偉成他們那種人,是活不過我的。倒是沒料到末了連他屍骨也沒收著。來到台灣,天天忙著過活,大陸上的事情,竟然逐漸淡忘了。老實說,要不是在新生社又碰見朱青,我是不會想起她來了的。

過了兩天,朱青果然差了一輛計程車帶張條子來接我去吃晚餐。原來朱青住在信義路四段,另外一個空軍眷屬區。那晚她還有其他的客人,是三個空軍小伙子,大概週未從桃園基地來台北度假的,他們也順著朱青亂叫我師娘起來,朱青指著一個白白胖胖,像個麵包似的矮子向我說:

「這是劉騷包,師娘,回頭你瞧他打牌時,那副狂骨頭的樣兒就知道了。」

那個姓劉的便湊到朱青跟前嬉皮笑臉的嚷道:

「大姐,難道今天我又撞著你什麼了?到現在還沒有半句好話呢。」

朱青只管吃吃的笑著,也不去理他,又指著另外一個瘦黑瘦黑的男人說:

「他是開小兒科醫院的,師娘只管叫他王小兒科就對了。他和我們打了這麼久的麻將,就沒和出一副體面的牌來。他是我們這裡有名的雞和大王。」

那個姓王的笑歪了嘴,說:

「大姐的話先別說絕了,回頭上了桌子,我和老劉上下手把大姐夾起來,看大姐再賭厲害。」

朱青把麵一揚,冷笑道:

「別說你們這對寶器,再換兩個厲害的來,我一樣有本事教你們輸得當了褲子才準離開這兒呢。」

朱青穿了一身布袋裝,肩上披著件紅毛衣,袖管子甩蕩甩蕩的,兩筒膀子卻露在外面。她的腰身竟變得異常豐圓起來,皮色也細緻多了,臉上畫得十分入時,本來生就一雙水盈盈的眼睛,此刻顧盼間,露出許多風情似的。接著朱青又替我介紹了一個二十來歲叫小顧的年輕男人。小顧長得比先頭那兩個體面得多,茁壯的身材,濃眉高鼻,人也厚實,不像那兩個那麼嘴滑。朱青在招呼客人的時候,小顧一徑跟在她身後,替她搬挪桌椅,聽她指揮,做些重事。

不一會,我們入了席,朱青便端上了頭一道菜來,是一盆清蒸全雞,一個琥珀色的大瓷碗裡盛著熱氣騰騰的一隻大肥母雞,朱青一放下碗,那個姓劉的便跳起來走到小顧身後,直推著他嚷道:

「小顧,快點多吃些,你們大姐燉雞來補你了。」

說著他便跟那個姓王的笑得發出了怪聲來。小顧也跟著笑了起來,臉上卻十分尷尬。朱青抓起了茶几上一頂船形軍帽,迎著姓劉的兜頭便打,姓劉的便抱了頭繞著桌子竄逃起來。那個姓王的拿起羹匙舀了一瓢雞湯送到口裡,然後舐唇咂嘴的嘆道:

「小顧來了,到底不同,大姐的雞湯都燉得下了蜜糖似的。」

朱青丟了帽子,笑得彎了腰,向那姓劉的和姓王的指點了一頓,咬著牙齒恨道:

「兩個小挨刀的,詔了大姐的雞湯,居然還吃起大姐的豆腐來!」

「大姊的豆腐自然是留給我們吃的了。」姓劉的和姓王的齊聲笑道。

「今天要不是師娘在這裡,我就要說出好話來了,」朱青走到我身邊,一隻手扶在我肩上笑著說道,「師娘,你老人家莫見怪。我原是召了這群小弟弟來侍候你老人家八圈的,哪曉得幾個小鬼頭平日被我慣壞了,嘴裡沒上沒下混說起來。

朱青用手戳了一下那個姓劉的額頭,說:

「就是你這個騷包最討人喜歡!」

說著便走進廚房裡去了。小顧也跟了進去幫朱青端菜出來。那餐飯我們吃了多久,姓劉的和姓王的便和朱青說了多久的風話。

自從那次以後,隔一兩個禮拜,朱青總要來接我到她家去一趟。可是見了她那些回數,過去的事情,她卻一句也沒提過。我們見了面總是忙著搓麻將。朱青告訴我說,小顧什麼都不愛,惟獨喜愛這幾張。他一放了假,從桃園到台北來,朱青就四處去替他兜搭子,常常連她巷子口那家雜貨店一品香老闆娘也拉了來湊腳。小顧和我們打牌的當兒,朱青便不入局,她總端張椅子,挨著小顧身後坐下,替小顧點張子。她蹺著腳,手肘子搭在小顧肩上,嘴裡卻不停的哼著歌兒,又是什麼《嘆十聲》,又是什麼《怕黃昏》,唱出各式各樣的名堂來。有時我們會打多久的牌,朱青便在旁邊哼多久的歌兒。

「你幾時學得這麼會唱歌了,朱青?」有一次我忍不住問她道,我記起她以前講話時,聲音都怕抬高些的。

「還不是剛來台灣找不到事,在空軍康樂隊裡混了這麼多年學會的。」朱青笑著答道。

「秦老太,你還不知道呀,」一品香老闆娘笑道,「我們這裡都管朱小姐叫‘賽白光’呢。」

「老闆娘又拿我來開胃了,」朱青說道,「快點用心打牌吧,回頭輸脫了底,又該你來鬧著熬通宵了。」

遇見朱青才是三、四個月的光景,有一天,我在信義路東門市場買滷味,碰見一品香的老闆娘在那兒辦貨,她一見了我就一把抓住我的膀子叫道:

「秦老太,你聽見沒有?朱小姐那個小顧上禮拜六出了事啦!他們說就在桃園的飛機場上,才起飛幾分鐘,就掉了下來。」

「我並不知道呀。」我說。

一品香老闆娘叫了一輛三輪車便和我一同往朱青家去看她去。一路上一品香老闆娘自說自話叨登了半天:

「這是怎麼說呢?好好的一個人一下子就沒了。那個小顧呀,在朱小姐家裡出入怕總有兩年多了。初時朱小姐說小顧是她幹弟弟,可是兩個人那麼眉來眼去,看著又不像。依百順,到哪裡去找? 我替朱小姐難過!

我們到了朱青家,按了半天鈴,沒有人來開門,不一會兒,卻聽見朱青隔著窗子向我們叫道:

「師娘,老闆娘,你們進來呀,門沒有閂上呢。」

我們推開門,走上她客廳裡,卻看見原來朱青正坐在窗台上,穿了一身粉紅色的綢睡衣,撈起了褲管蹺腳,在腳趾甲上塗寇丹,一頭的發捲子也沒有卸下來。她看了我們抬起頭笑道:

「我早就看見你們兩個了,指甲油沒乾,不好穿鞋子走出去開門,叫你們好等——你們來得正好,晌午我才燉了一大鍋糖醋蹄子,正愁沒人來吃。

正說著餘奶奶便走了進來。朱青慌忙從窗台上跳了下來,收了指甲油,對著一品香老闆娘說:

「老闆娘,煩你替我擺擺桌子,我進去廚房端菜來。今天都是太太們,手腳快,吃完飯起碼還有二十四圈好搓。」

朱青進去廚房,我也跟了進去幫個忙兒。茱青把鍋裡的糖醋蹄倒了出來,又架上鍋頭炒了一味豆腐。我站在她身旁端著盤子等著替她盛菜。

「小顧出了事,師娘該聽到了?」朱青一邊炒菜,頭也沒有回,便對我說道。

「剛才一品香老闆娘告訴我了。」我說。

「小顧這裡沒有親人。他的後事由我和他幾個同學料理清楚了。昨天下午,我才把他的骨灰運到碧潭公墓下了葬。」

我站在朱青身後,瞅著她,沒有說話,朱青臉上沒有施脂粉,可是看著還是異樣的年輕朗爽,全不像個三十來歲的婦人,大概她的雙頰豐腴了,肌膚也緊滑了,歲月在她的臉上好像刻不下痕跡來了似的。我覺得雖然我比朱青還大了一大把年紀,可是我已經找不出什麼話來可以開導她的了。朱青俐落的把豆腐兩翻便起了鍋,然後舀了一瓢,送到我嘴裡,笑著說:

「師娘嚐嚐我的‘麻婆豆婆’,可夠味了沒有?」

我們吃過飯,朱青便擺下麻將桌子,把她待客用的那副蘇州竹子牌拿了出來。我們一坐下去,頭一盤,朱青便撂下一副大三元來。

「朱小姐,」一品香老闆娘嚷道,「你的運氣這樣好,該去買‘愛國獎券’了!」

「你們且試著吧,」朱青笑道,「今天我的風頭又要來了。」

八圈上頭,便成了三歸一的局面,朱青面前的籌碼堆到鼻尖上去了。朱青不停的笑聲,嘴裡翻來滾去哼著她常愛唱的那首《東山一把青》。隔不了一會兒,她便哼出兩句:

噯呀噯噯呀,

郎呀,採花兒要趁早哪——

% % 设置目录中附录标题为繁体中文
% \renewcommand{\appendixname}{附錄}
% \appendix
% \chapter{這是附錄A}

% \backmatter
% This is \textbackslash backmatter

\fussy