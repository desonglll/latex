\section{上部}

\sloppy

抗日勝利,還都南京的那一年,我們住在大方巷的仁愛東村,一個中下級的空軍眷屬區。在四川那種閉塞的地方,煎熬了那些年數,驟然回返那六朝金粉的京都,到處的古蹟,到處的繁華,一派帝王氣象,把我們的眼睛都看花了。

那時偉成正擔任十一大隊的大隊長。他手下有兩個小隊剛從美國受訓回來,他那隊飛行員頗受重視,職務也就格外繁忙。遇到緊要差使,常由他親自率隊出馬。一個禮拜,倒有三、四天,連他的背影兒我都見不著。每次出差,他總帶著郭軒一起去。郭軒是他的得意門生,郭軒在四川灌縣航校當學生的時候,偉成就常對我說:郭軒這個小伙子靈跳過人,將來必定大有出息。果然不出幾年,郭軒便竄了上去,爬成小隊長留美去了。

郭軒是空軍的遺族。他父親是偉成的同學,老早摔了機,母親也跟著病歿了。在航校的時候,逢年過節,我總叫他到我們家吃餐團圓飯。偉成和我膝下無子,看著郭軒孤單,也常照顧他些。那時他還剃著青亮的頭皮,穿了一身土黃布的學生裝,舉止雖然處處露著聰明,可是口角到底嫩稚,還是個未經世的後生娃仔。當他從美國回來,跑到我南京的家來,衝著我倏地敬個軍禮,叫我一聲師娘時,我著實吃他唬了一跳。郭軒全身都是美式凡立丁的空軍制服,上身罩了一件翻領鑲毛的皮夾克,腰身勒得緊峭,褲帶上卻繫著一個Rav-Ban太陽眼鏡盒兒。一頂嶄新高聳的軍帽帽沿正壓在眉毛上;頭髮也蓄長了,滲黑油亮的髮腳子緊貼在兩鬢旁。才是一兩年工夫,沒料到郭軒竟出挑得英氣勃勃了。

「怎麼了,小伙子?這次回來,該有些苗頭了吧?」我笑著向他說。

「別的沒什麼,師娘,倒是在外國攢了幾百塊美金回來。」郭軒說。

「夠討老婆了!」我笑了起來。

「是呀,師娘,正在找呢。」郭軔也朝著我齜了牙齒笑道。

戰後的南京,簡直成了我們那些小飛行員的天下。無論走到哪裡,街頭巷尾,總碰到個把趾高氣揚的小空軍,手上挽了個衣著人時的小姐,瀟瀟灑灑,搖曳而過。談戀愛-個個單身的飛行員都在談戀愛。一個月我總收得到幾張偉成學生送來的結婚喜帖。可是郭軒從美國回來了年把,卻一直還沒有他的喜訊。他也帶過幾位摩登小姐到我家來吃我做的豆瓣鯉魚。事後我問起他,他總是搖搖頭笑著說:

「沒有的事,師娘,玩玩罷了。」


但有一天,他卻跑來告訴我:這次他認了真了。他愛上了一個在金陵女中念書叫朱青的女孩兒。

「師娘,」他一股勁的對我說道,「你一定會喜歡她,我要帶她去見你。師娘,我從來沒想到會對一個女孩子這樣認真過。」

郭軒那個人的性格,我倒摸得著一二。心性極為高強,年紀輕,發跡早,不免有點自負。平常談起來,他曾對我說,他必得要選中一個稱心如意的女孩兒,才肯結婚。他帶來見我的那些小姐,個個容貌不凡,他都沒有中意,我私度這個朱青大概是天仙一流的人物,才會使得郭軒如此動心。

當我見到朱青的時候,卻大大的出了意料之外。那天郭軒帶她來見我,在我家吃午餐。原來朱青卻是個十八九歲頗為單瘦的黃花閨女,來做客還穿著一身半新舊直統子的藍布長衫,襟上掖了一塊白綢子手絹兒。頭髮也沒有燙,抿得整整齊齊的垂在耳後。腳上穿了一雙帶絆的黑皮鞋,一雙白色的短統襪子倒是乾乾淨淨的。我打量了她一下,發覺她的身段還未出挑得周全,略略扁平,面皮還泛著些青白。但她的眉眼間卻蘊著一脈令人見之忘俗的水秀,見了我一頭半低著頭,靦靦腆腆,很有一股教人疼憐的怯態。一頓飯下來,我怎麼逗她,她都不大答得上腔來,一味含糊的應著。倒是郭軒在一旁卻著了忙,一忽兒替她拈菜,一忽兒替她斟茶,直慫著她跟我聊天。


「她這個人就是這麼彆扭,」郭軒到了後來急躁的指著朱青說道,「她跟我還有話說,見了人卻成了啞巴。師娘這兒又不是外人,也這麼出不得眾。」

郭軫的話語說得暴躁了些,朱青扭過頭去,羞得滿面通紅。

「算了,」我看著有點不過意,忙止住郭軒道,「朱小姐頭一次來,自然有點拘泥,你不要去戳她。吃完飯還是你們兩人去遊玄武湖去罷,那兒的荷花開得正盛呢。

郭軒是騎了他那輛十分招搖的新摩托車來的。吃完飯,他們離開的時候,郭軒把朱青扶上了後車座,幫著她繫上她那塊黑絲頭巾,然後跳上車,輕快的發動了火,向我得意洋洋的揮了揮手,倏地一下,便把朱青帶走了。朱青偎在郭軒身後,頭上那塊絲中吹得高高揚起。看著郭軒對朱青那副笑容,我知道他這次果然認了真了。

有一次,偉成回來,臉色沉得很難看,一進門便對我說:

「郭軒那小伙子越來越不像話!我倒沒料到他竟是這樣一個人」

「怎麼了?」我十分詧異,我從來沒有聽見偉成說過郭詔一句難聽的話。

「你還問得出呢!你不是知道他在追一個金陵女中的學生嗎?我看他這個人談戀愛談昏了頭!經常闖進人家學校裡去,也不管人家在上課,就去引誘那個女學生出來。我們總部來了,成個什麼體統?

郭軫被記了過,革除了小隊長的職務。當我見到郭軒時,他卻對我解說:

「師娘,不是我故意犯規,惹老師生氣,是朱青把我的心拿走了。真的,師娘,我在天上飛,我的心都在地上跟著她呢。朱青是個規規矩矩的好女孩,就是有點怕生,不大會交際罷了。定了我,現在她一個人住在一間小客棧裡還沒有著落呢。

「傻子,」我搖頭嘆道,沒想到聰明人談起戀愛來,也會變得這般糊塗,「既是這麼痴,兩人結婚算了。」

「師娘,我就是要來和你商量這件事,要請你和老師做我們的主婚人呢。」郭軒滿面光彩對我說。

郭軒和朱青結婚以後,也住在我們仁愛東村。郭軒有兩個禮拜的婚假,本來他和朱青打算到杭州去度蜜月的,可是還沒去成,猛然間國內的戰事便爆發了。偉成他們那支大隊被調到東北去。臨走的那天早上,才濛濛亮,郭軒便鑽進我的廚房裡來,我正在升火替偉成煮泡飯。郭軒披著件軍外套,頭髮蓬亂,兩眼全是紅絲,鬍鬚也沒剃,一把攥住我手,嗓子嘎啞,對我說:

「師娘,這次無論如何要拜託你老人家了——」

「曉得了,」我打斷他的話道,「你不在,自然是我來照顧你老婆啦。」

「師娘——」郭軒還在嘮叨,「朱青還不大懂事,我們空軍裡的許多規矩,她不甚明了,你要當她自己人,多多教導她才好。」

「是了,」我笑道,「你師娘跟著你老師在空軍裡混了這十來年,什麼還沒見過?不知多少人從我這裡學了乖去呢。朱青又不笨,你等我來慢慢開導她。


偉成和郭軒他們離開後,我收拾了一下屋子便走到朱青家去探望她。公家配給郭軒他們的宿舍是一幢小巧的木板平房。他們搬進去以前,郭軒特別找人粉刷過一輪,掛上些新的門簾窗幔,相當起眼。我進到他們的房子裡,看見客廳裡還是新房般的打扮。桌子椅子上堆滿了紅紅綠綠的賀禮,有些包裹尚未拆封。桌子跟下卻圍著一轉花籃,那些玫瑰劍蘭的花苞兒開得十分新鮮,連鳳尾草也是碧綠的。牆上那些喜幛也沒有收去,郭軒同學送給他的一塊烏木燙金的喜匾卻懸在廳的中央,寫著「白頭偕老」。

朱青在她房裡,我走進去她也沒聽見。她歪倒在床上,臉埋在被窩裡,抽抽搭搭的哭泣著。她身上仍舊穿著新婚的艷色絲旗袍,新燙的頭髮揉亂了,髮尾子枝椏般生硬的張著。一床繡滿五彩鴛鴦的絲被面被她搓得全是皺紋。在她臉旁被面上,卻浸著一塊碗大的濕印子。她聽見我的腳步驚坐了起來,只叫出一聲「師娘」,便只有哽咽的份兒了。朱青滿面青黃,眼睛腫得瞇了起來,看著愈加瘦弱了。我走過去替她抿了一下頭髮,絞了一把熱手巾遞給她。朱青接過手中,把臉摀住,重新又哭泣起來。房子外頭不斷的還有大卡車和吉普車在拖拉行李,鐵鍊鐵條撞擊的聲音,非常刺耳,村子裡的人正陸續啟程上任,時而女人尖叫,時而小孩啼哭,顯得十分惶亂。我等朱青哭過了,才拍拍她的肩膀說:

「頭一次,乍然分離,總是這樣的——今晚不要開夥,到我那兒吃夜飯,給我做個伴兒。」

偉成和郭軒他們一去便了無蹤跡。忽而聽見他們調到華北,忽而又來信飛到華中去了,幾個月來一次也沒回家。這個期間,朱青常常和我在一起。有時我教她做菜,有時我教她織毛衣,也有時我教她玩幾張麻將牌。

「這個玩意兒是萬靈藥,」我對她笑著說道,「有心事,坐上桌子,紅中白板一混,什麼都忘了。」

朱青結婚後,放得開多了,可是仍舊靦腆怯生,除掉我這兒,村子裡別家她一概沒有來往。村子裡那些人的身世我都知曉,漸漸兒的,我也揀了一些告訴她聽,讓她熟悉一下我們村裡那些人的生活。

「你別錯看了這些人,」我對她說,「她們背後都經過了一番歷練的呢。像你後頭那個周太太吧,她已經嫁了四次了。她現在這個丈夫和她前頭那三個原來都是一個小隊裡的人。先生原是她小叔,徐家兩兄弟都是十三大隊裡的。

「可是她們看著還有說有笑的。」朱青望著我滿面疑惑。

「我的姑娘,」我笑道,「不笑難道叫她們哭不成?要哭,也不等到現在了。」

郭軫離開後,朱青一步遠門也不肯出,天天守在村子裡。有時我們大夥兒上夫子廟去聽那些女孩們清唱,朱青也不肯跟我們去。她說她怕錯過總部打電話傳來郭軒的消息,一天夜裡,總部帶信來說,偉成那一隊經過上海,有一天多好停留,可能趕到南京來。朱青一早就跳出跳進,忙著出去買了滿滿兩籃菜回來。下午我經過她門口,看見她穿了一身藍布衣褲,頭上繫了一塊舊頭巾,站在凳子上洗窗戶。她人又矮小,踮起腳還夠不著,手裡卻揪住一塊大抹布揮來揮去,全身的勁都使出來了似的。

「朱青,那上頭的灰塵,郭軒看不見的。」我笑著叫道。

朱青回頭看見我,紅了臉,訕子的說:

「不知怎的,才幾個月,這問房子便舊了,洗也洗不乾淨。」

傍晚的時分,朱青過來邀請了我一塊兒到村口擱軍用電話的那間門房裡去等候消息。總部那邊的人答應六、七點鐘打電話給我們。朱青梳洗過了,換上一件杏黃色的薄綢長衫,頭上還絡了一根蘋果綠的絲帶,嘴上也抹了一些口紅,看著十分清新可喜。起初朱青還非常開心,跟我有說有笑,到了六點多鐘的光景,她便漸漸緊張起來了,臉也繃了,聲也噤了,她一邊織著毛線卻不時的抬頭去看桌上那架電話機。我們左等右等,直到九點多鐘,電話鈴才響了起來。朱青倏地跳起來,懷裡的絨線球滾到一地,急忙向電話奔去,可是到了桌子邊卻回過頭來向著我聲音顫抖的說道:

「師娘——電話來了。」

我去接過電話,總部裡的人說,偉成他們在上海只待了兩小時,下午五點鐘已經起飛到蘇北去了。我把這個消息告訴朱青,朱青的臉色一下子變得非常難看,她呆站著,半晌沒有出聲,臉上的肌肉卻微微的在抽搐。

「我們回去吧。」我向她說。

我們走回村子裡,朱青一直默默跟在我後面,走到我家門口時,我對她說:

「莫難過了,他們的事情很沒準的。」

朱青扭過頭去,用袖子去擂眼睛,嗓子哽咽得很厲害。

「別的沒有什麼,只是今天又空等了一天——」

我把她的肩膀摟過來說:

「朱青,師娘有幾句話想跟你講,不知你要不要聽。飛將軍的太太,不容易當。廿四小時,那顆心都掛在天上,哪怕你眼睛朝天望出血來,那天上的人未必知曉。你就得狠起心腸來,才擔得住日後的風險呢。

朱青淚眼模糊的瞅著我,似懂非懂的點著頭兒。我扳起她的下巴頦,笑著嘆道:

「回去吧,今夜早點上床。」

民國三十七年的冬天,我們這邊的戰事已經處處失利了,北邊一天天吃緊的當兒,我們東村里好幾家人都遭了兇訊。有些眷屬天天到廟裡去求神拜菩薩,算命的算命,摸骨的摸骨。我向來不信這些神鬼鬼,偉成久不來信,我便邀隔壁鄰舍來成桌牌局,熬個通宵,定定神兒。有一晚,我跟幾個鄰居正在斗牌兒,住在朱青對門的那個徐太太跑來一把將我拖了出去,上氣不接下氣的告訴我說總部剛來通知,郭軒在徐州出了事,飛機和人都跌得粉碎。我趕到朱青那兒,裡面已經黑壓擠擠滿了一屋子的人。朱青歪倒在一張靠椅上,左右一邊一個女人揪住她的膀子,把她緊緊按住,她的頭上紮了一條白毛巾,毛巾上紅殷殷的沁著巴掌大一塊血跡。我一進去,裡面的人便七嘴八舌告訴我:朱青剛才一得到消息,便抱了郭軒一套制服,往村外跑去,一邊跑一邊嚎哭,口口聲聲要去找郭軒。有人攔她,她便亂踢亂打,剛跑出村口,便一頭撞在一根鐵電線桿上,額頭上碰了一個大洞,剛才抬回來,連聲音都沒有了。

我走到朱青跟前,從別人手中接過一碗薑湯,用銅羹匙撬開朱青的牙關,紮實的灌了她幾口。她的一張臉像是劃破了的魚肚皮,一塊白,一塊紅,血汗斑斑。她的眼睛睜得老大,目光卻是散渙的。她沒有哭泣,可是兩片發青的嘴唇卻一直開合著,喉頭不斷發出一陣陣尖細的聲音,好像一隻瞎耗子被人踩得發出吱吱的慘叫來一般。我把那碗薑湯灌完了,她才漸漸的收住目光,有了幾分知覺。


朱青在床上病了許久。我把她挪到我屋子裡。日夜守住她,有時連我打牌的時候,也把她放在跟前。我怕走了眼,她又去尋短見。朱青整天睡在床上。也不說話,也不吃東西。每天都由我強灌她一點湯水。幾個禮拜,朱青便瘦得只剩下了一把骨頭,面皮死灰,眼睛凹成了兩個大窟窿。有一天我餵完她,便坐在她床邊,對她說:

「朱青,若說你是為了郭軒,你就不該這般作踐自己。就是郭軒在地下,知道了也不能心安哪。」

朱青聽了我的話,突然顫巍巍的掙扎著坐了起來,朝我點了兩下頭,冷笑道:

「他知道什麼?他跌得粉身碎骨哪裡還有知覺?他倒好,轟地一下便沒了——我也死了,可是我卻還有知覺呢。」

朱青說著,面上似哭似笑的扭曲起來,非常難看。

守了朱青個把月,自己都差不多累了。幸而她老子娘卻從重慶趕了來。她老子看見她一句話都沒有說,她娘卻狠狠的啐了一口:

「該呀!該呀!我要她莫嫁空軍,不聽話,落得這種下場!」

說著便把朱青蓬頭垢面的從床上扛下來,用板車連鋪蓋一齊拖走了。朱青才走幾天,我們也開始逃難,離開了南京。

\fussy