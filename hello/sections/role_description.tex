\section*{角色簡介}
\addcontentsline{toc}{section}{角色簡介} % 加入到目錄
% 定義新的 \customsubsection 命令
\newcommand{\csubsection}[1]{
    % 增加 subsection 计数器值 1
    % \addtocounter{subsection}{1}
    \subsection*{#1} % 取消編號的標題
    \addcontentsline{toc}{subsection}{#1} % 加入到目錄
}
\csubsection{秦芊儀}

人稱師娘,華南師範大學英文科退學,家族為浙江望族,抗戰前夕,嫁給十一大隊江偉成而放棄學業。外柔內剛的師娘,一路陪伴著偉成,並打理著空軍村裏的一切。她的青春,化作成跑道盡頭的行燈,在大時代裡,微弱地閃爍著……她展現了那個年代女人的樣貌,溫柔卻剛強,為了愛情,不顧家人反對堅持嫁給偉成,從此過著平凡、但沒有安全感的生活,卻能以智慧和堅韌態度面對一切,一路照顧與開導,帶領女人們繼續向前,曾經一度被認定是匪諜而入獄,後來被釋放,最後留在空軍村過完一生。

\csubsection{江偉成}

空軍中校大隊長,飛行技術精湛,性格剛毅果斷並嚴賞罰。從抗戰勝利、眾所景仰的英雄時的意氣風發,到為了陪伴妻子,考慮轉參謀職的內心交戰,至內戰爆發,指揮作戰,失去大隊所有飛機和飛行員之後,內心自責與後半生的痛苦,而後輾轉到台灣後過著清貧生活。晚景的淒涼與無奈,是整個動盪時代下,諸多戰爭老兵的象徵人物,最後為了不要拖累芊儀上吊自殺。原著中在海上染病死亡,沒有跟隨秦芊儀來到台灣。


\csubsection{周瑋訓}

小周,空軍遺屬軍眷,人稱副隊娘。父為前清武舉人,家族為遼北省望族。抗戰時期就讀於華南師範大學體育科,在校時與師娘結為手帕交。其夫原為第十一大隊副隊長,對日抗戰時殉職,之後依著空軍傳統,學長陣亡,托妻學弟,改嫁小邵。性格直率的她,常用她的大嗓門,嬉笑怒罵的掩飾內心荒寂。離亂歲月,每在時代嘯浪裡安然漂出的她,祇能任著師娘及朱青漸漸偏離了人生航線,無能為力……小周的故事最能代表空軍特殊「傳統」的角色,面對隨時可能發生的死亡與別離,在堅強中掩藏著脆弱。她的個性率真,情感卻細膩,活潑中帶點潑辣的個性,讓有她在的場合總能帶來溫暖與歡笑。口頭禪:「狗肉進不了大上海」,後來搬離空軍村。原著中僅透過芊儀的話出場。

\newpage

\csubsection{邵志堅}

小邵,第十一大隊少校副隊長,到台灣後為十一大隊中校大隊長。個性正直、謹慎,遇事總瞻前顧後,也因如此,總在需要果斷抉擇時錯失良機。最初在南京原任大隊上尉作戰官,因原任副隊長代其迎戰日軍身亡,情義使然,放棄了與女友的感情,依空軍傳統和同袍遺願,「交接」小周與墨婷母女,自此生活在責任與摯愛的兩難之中,最後在台灣因匪諜案牽連,從十一大隊中校大隊長降調空軍工程隊少校隊長。


\csubsection{郭軫}

原就讀西北大學工程系航空工程組,對日抗戰時慨然投筆從戎,考入中央航校,第十一大隊中尉分隊長,是偉成的得意門生,對同袍有情有義的他,也因空中作戰的命懸一線,笑看人生,玩世不恭。作戰負傷,於浙江聯中女生宿舍養傷之際,留下祇有編號未有姓名的字條。未料青春少艾的朱青竟依字條千里尋來。
漂泊靈魂終有自己的導航塔,他與朱青產生一段轟轟烈烈、刻骨銘心,但卻短暫的愛情。


\csubsection{朱青}

一個青澀杭州師範女學生,帶著少女對愛情的渴望,因為一張空軍飛行員郭軫留下的字條來到南京。原本,她祇好奇那留字條的飛行員長什麼樣子,但千里見得郭軫後,她中斷了金陵女子大學的學業,與郭軫相戀,嫁進空軍眷村。然而最後等待她的,只有失去座標的人生。樸素、純潔的女學生,經歷了慘痛的生離死別而後變得成熟世故,判若兩人。心死之後,卻以玩世態度面對人生,麻木不仁,再也沒有什麼事可以傷害她,曾經一度被認定是匪諜而入獄,後來跟芊儀互換而出獄,因顧肇鈞幫助而實現了到美國定居的願望,最後回到台灣跟墨婷在前空軍機棚重逢。


\csubsection{顧肇鈞}

小顧,菜鳥飛官,抗戰末期考入航校。畢業後,對日抗戰已結束,往各大隊任准尉見習官見習。均被考評為不適飛行。小顧個性執拗,堅持要當飛行員。上級給了最後機會-調入十一大隊作最後考評。來到十一大隊後遇見已嫁給郭軫的朱青。此時小顧發現繫起朱青、郭軫千里姻緣的字條,自己才是原來的作者,自以為自己才是字條主人,甘冒大不韙,執著於對學長妻子朱青的愛戀。到台灣之後再見朱青,對歷經滄桑的她更加憐惜,以自己的方式表達對朱青的愛,在最後任務中失蹤(疑似被擊落)。


\csubsection{靳墨婷(邵墨婷)}

編劇黃世鳴以旁觀者的手法,讓墨婷時而成為劇中人,又時而成為描述劇情的旁白者。她聰慧早熟,年紀雖小,卻看盡了那段離亂的時代-在母親及兩個乾媽的身邊,親眼目睹那場戰爭的震盪,對男人、對女人、對一切生命的,無情催逼。生父曾是飛行大隊副隊長,在抗戰時期遭日軍擊落而殉職,後因母親小周改嫁,成為一位擁有兩位父親姓氏的女孩。之後隨部隊撤退到台灣,展開了她自己的青春,和焦飛開展一段屬於眷村第二代的生命故事,也看盡了空軍村裡的悲歡離合。