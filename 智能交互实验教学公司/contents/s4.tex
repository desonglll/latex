\subsection{北京奕斯伟计算技术股份有限公司}\label{北京奕斯伟计算技术股份有限公司}

\subsubsection{项目内容}

\subsubsubsection{教育信息化进程}
教育信息化从1.0时代迈进2.0时代,教育信息建设正从设备、平台的建设转变至以人为本的信息化建设。以人工智能计算为代表的新一代技术不断地促进教师“教”的转变和学生“学”的改革,推动传统的学校教育从统一化、规模化转向定制化、个性化。

\subsubsubsection{奕斯伟智慧教育解决方案}
奕斯伟智慧教育解决方案由教育边缘智能站、设备云平台及配套的信息化设施共同构成,面向教育中教、学、评、测、考、管等场景,搭载各类教育算法,提供智慧教育的整体解决方案,实现信息的高效实时分析、资源共享和互动增强,助力教育智能化、信息化发展。

\subsubsection{方案全景图}

\subsubsubsection{教育边缘智能站+设备云平台}
奕斯伟教育边缘智能站(教育小脑)是一款面向教育场景的边缘计算终端,具有超强音视频编解码及智能化处理能力,搭载实时教育音视频交互、教学智能跟踪及导播、教学行为分析、手写板书提取、理化生实验算法等等人工智能算法,可广泛应用于互动教学、智能实验、课堂分析、智慧体育等场景。设备云平台提供高可靠性的集中存储、实时管理、远程OTA升级及运维等服务,与边缘智能站实现云边协同,形成智慧教育解决方案的核心。

\subsubsubsection{方案优势}
\begin{itemize}
    \item \textbf{边缘计算,实时分析:} 依托边缘算力,实现对音频、视频的高效实时处理与分析。
    \item \textbf{集约化、一体化设计:} 采用先进的智能计算芯片,结合先进的音视频AI处理技术,具备音视频一体化智能处理能力。
    \item \textbf{快速开发,快速迭代:} 内置基础算法及AI开发框架,简化开发流程、缩短开发周期,实现快速开发迭代。
    \item \textbf{算法丰富,覆盖全面:} 具备回声消除、自动增益、背景降噪、啸叫抑制等音频算法;学生、老师智能跟踪及AI导播算法;教学行为分析算法;理化生智慧实验算法等多个场景AI算法。
    \item \textbf{云边协同,远程运维:} 远程设备管理、OTA升级,算法远程更新,有效降低运维成本.
\end{itemize}

\subsubsection{应用场景}

\subsubsubsection{智能实验}
通过对实验器材、实验流程、实验操作关键细则进行算法深度学习,对初高中的物理、化学及生物实验学习过程进行实时动态跟踪评测,可完整记录学生的实验数据,追溯考试过程,提供更客观公正的AI智能评分。

\subsubsubsection{智能课堂分析}
对课堂教学过程无感采集,对课堂环境及氛围、教学内容等多维分析,精准识别包括师生互动、讲授、问答、讨论、对话、评价、练习、笔记、纪律干预等教学行为,为课堂教学质量提升、实证化教研、教学管理提供全面科学的数据支撑。

\subsubsubsection{智能互动教学}
实现主讲教室与互动教室、听课教室同上一节课,老师与远程教室的学生实时互动,满足各种网络条件下教学、教研、管理等业务的开展,突破时间地域限制,让优质教育普及范围更广。

\subsubsection{相关技术}

\subsubsubsection{智能计算SoC}
奕斯伟智能计算SoC,采用64位乱序执行RISC-V处理器,搭载自研高能效NPU,支持H.264、H.265视频编解码标准,具备强大的音视频处理能力,芯片算力最高可达39.9TOPs。支持全栈浮点计算,全面加速生成式大模型,有效解决AI PC、工业视觉等行业的应用难题。产品可根据用户需求提供芯片、开发板或边缘智能站等多种形态,为客户在大语言模型、语音合成、图像/视频生成、文本生成等领域提供高效能智能计算解决方案。


\subsection{鸿合科技股份有限公司}

\subsubsection{智慧教育服务}
\subsubsubsection{研发中心}
智慧教育服务涵盖了多种创新解决方案,旨在通过先进技术提升教育质量和效率。研发中心专注于开发和优化这些解决方案,以满足不同教育场景的需求。

\subsubsubsection{K12智慧教学}
K12智慧教学方案结合了最新的技术手段和教育理念,为K12教育阶段的学生提供了个性化、互动式的学习体验。这些方案包括智能互动设备和专属教学软件,能够支持各种教学模式和学习风格。

\subsubsubsection{鸿合智慧教室解决方案}
鸿合智慧教室解决方案以其先进的技术和全面的功能,致力于为学校提供高效的教学支持。该方案涵盖了从智能黑板到交互平板的多种设备,支持各种教学需求和场景。

\subsubsection{鸿合三个课堂解决方案}
\subsubsubsection{高职教全场景互联应用}
通过鸿合高职教智能交互黑板和智能交互平板,结合专属的交互教学软件,打造了全面的高职教互动教学解决方案。该方案支持双屏教学、多屏互动和小组协作,能够有效提升课堂的吸引力和学生的学习积极性。它适用于小班化教学、混合式教学和翻转课堂等多种新型教学模式,为高职院校建设现代化互动教室提供了有力的支持。

\subsubsubsection{幼教智能学习空间}
幼教智能学习空间方案专注于为幼儿教育阶段提供智能化的学习环境。通过整合各种智能设备和教学工具,营造出一个互动、有趣的学习空间,帮助幼儿在玩乐中学习,促进他们的全面发展。

\subsubsubsection{方案综述}
鸿合的高职教智能交互黑板(包括红外和电容两种型号)和智能交互平板,结合多种教学应用和工具,形成了一个综合的教学解决方案。该方案支持常态化的小班教学和多种互动模式,如双屏教学和研讨型混合教学,能够有效增强课堂互动性和教学效果。同时,沉浸式教学解决方案提供了身临其境的学习体验,支持远程教学和同声传译,便于沟通和知识传递。

\subsubsection{核心优势}
\subsubsubsection{常态化小班教学}
高职教智能交互设备支持常态化的小班教学模式,通过多种应用和互动工具,激活课堂氛围,提升教学效果。

\subsubsubsection{双屏教学}
双屏教学模式允许主辅屏结合,提升教学效果,并加强记忆关联。双屏的应用提供了更多的展示和互动方式,增加了课堂的互动性和参与度。

\subsubsubsection{研讨型混合教学}
该方案支持多人协作和实时反馈,适用于小组研讨型教学环节,能够有效促进学生之间的互动和讨论。

\subsubsubsection{沉浸式教学解决方案}
沉浸式教学解决方案通过身临其境的学习体验、远程教学和同声传译系统,提供了一个全面的沟通和学习平台,提升了教学的便捷性和效果。


\subsection{阿凡达高等院校AI教育}

\subsubsection{产品矩阵}
阿凡达提供了一系列高职高校实训室解决方案,旨在满足日常实训和职业教育专业教学的需求。这些解决方案包括人工智能、无人驾驶、移动机器人和HarmonyOS等多个领域,旨在有效提升实训教学质量和效率,并创新数字化教育和教学模式。

\subsubsection{合作方向}
\subsubsubsection{实训室共建}
阿凡达致力于贯穿研发、产业落地和产业生态建设,以全球领先的人工智能和机器人核心技术为基础,结合产业元素,推进高等教育和职业教育的校内外人才培养。通过以下实训室的共建,提升学生的实操能力和专业素养:

\subsubsubsection{人工智能实训室}
该实训室用于核心人工智能课程的实验和实训,涉及语音识别、计算机视觉、自然语言处理和深度学习等技术,为学生提供全面的人工智能实践体验。

\subsubsubsection{嵌入式开发实训室}
该实训室针对嵌入式系统相关课程,涵盖单片机基本原理、输入输出接口、嵌入式处理器和嵌入式系统等核心知识点,帮助学生掌握嵌入式系统的开发技能。

\subsubsubsection{物联网实训室}
结合物联网产业技术,该实训室涉及传感器原理、物联网通讯协议、数据采集传输和系统搭建等内容,提供了实际操作物联网技术的机会。

\subsubsubsection{鸿蒙创新实训室}
专注于鸿蒙操作系统的操作原理和相关开发,该实训室支持在智能手机、智能穿戴、智能家居和智能座舱等领域进行创新实践。

\subsubsubsection{大模型综合实训室}
该实训室以低门槛、高效率的方式,让学生体验和训练大模型产品,实现基于大模型的专业教学升级。

\subsubsubsection{智能机器人实训室}
专为人工智能及机器人专业的创新实践教学而设计,结合人工智能技术与机器人技术,提升学生的实践能力和创新思维。

\subsubsubsection{工业4.0实训室}
助力培养工业互联网和智能制造专业技能人才,通过模拟企业级产线场景,提升学生的实际解决问题能力。

\subsubsubsection{无人驾驶实验室}
该实验室模拟智能汽车和无人驾驶的真实技术场景,让学生从硬件到软件、算法控制等方面进行全面实践,提升汽车产业IT技术能力。

\subsubsection{专业共建}
\subsubsubsection{产业学院共建}
阿凡达从产业用人需求出发,与院校共同规划专业人才培养体系,创新岗课证赛融通的人才培养模式,打造专业品牌标杆。面向区位经济发展及泛AI产业人才需求,校企共建特色产业学院,打造高水平专业群体。

\subsubsection{解决方案建设内容}
阿凡达结合在产业领域的技术研发、产品应用和落地经验,提供全融通的特色解决方案。这些解决方案涵盖“岗课证赛”各个方面,与合作伙伴共同建设融合互补的教育生态圈,内容包括课程、产品、师资培训、科研创新和实习就业等。


\subsection{西南交通大学与北京润尼尔科技股份有限公司}

\subsubsection{平台背景}

西南交通大学在数字教育背景下,紧跟国家数字化转型政策,探索虚拟仿真体系化、标准化、装备化建设实践,与北京润尼尔科技股份有限公司合作,于2023年建成了“智能虚拟仿真实验教学中心”。该中心旨在打造一个综合性的实验中心,涵盖智能虚拟仿真中心、实验教学中心、学生研创中心、实践教育基地、产教融合基地、未来学习中心、课程数字化资源开发中心、智能实验教学模式探索中心、技术展示基地、科普教育基地等多功能模块。

中心建设的总目标是“紧跟学科发展、创新人才培养模式,改革教学方式、打造沉浸教学体验,深化产教融合、促进校企深度合作,营造标杆效应,形成区域示范特色”。通过数智赋能和创新驱动,中心致力于打造一个全方位的“开放式”育人环境。

虚拟仿真实验教学是高等教育信息化建设和实验教学示范中心建设的重要内容,它体现了学科专业与信息技术的深度融合。为贯彻落实《教育部关于全面提高高等教育质量的若干意见》(教高〔2012〕4号)精神,并根据《教育信息化十年发展规划(2011-2020年)》的要求,教育部自2013年启动了国家级虚拟仿真实验教学中心的建设工作。虚拟仿真实验教学的管理和共享平台是这些中心建设中的重要组成部分。

当前,大多数高校虽然已经采购了各类实验教学软件,但由于各专业和课程的不同,这些软件的工作环境、体系结构、编程语言及开发方法各异。由于学校管理工作的复杂性,各校甚至校内各专业的实验教学建设往往各自为政,形成了“信息孤岛”。主要问题包括:

\begin{itemize}
    \item 管理混乱,各种实验教学软件缺乏统一的集中管理;
    \item 使用不规范,缺乏统一的操作模式和管理方式;
    \item 可扩展性差,无法支持课程和实验的扩展;
    \item 各系统的数据无法共享,容易形成“信息孤岛”;
    \item 缺乏足够的开放性;
    \item 软件部署复杂,不同的软件无法在同一台服务器上运行。
\end{itemize}

\subsubsection{平台目标}
该平台的主要目标是高效管理实验教学资源,实现校内外、本地区及更广范围内的实验教学资源共享。平台旨在统一接入所有实验软件,并实现学生在平台下进行统一实验的目标。通过系统间的无缝连接,达到整体的实验效果,解决信息孤岛问题,并支持学校快速实施第三方实验教学软件。

平台提供了全方位的虚拟实验教学辅助功能,包括门户网站、实验前的理论学习、实验开课管理、典型实验库的维护、实验教学安排、实验过程智能指导、实验结果自动批改、实验成绩统计查询、在线答疑、实验教学效果评估等。同时,平台可扩展集成第三方虚拟实验课程资源或自建课程资源,为各类院校虚拟实验教学环境提供服务和应用支持。

\subsubsection{应用方案}
根据《国家中长期教育改革和发展规划纲要(2010-2020年)》第十九章第六十条的要求,加强优质教育资源的开发与应用,建设开放灵活的教育资源公共服务平台,促进优质教育资源的普及共享。虚拟实验建设理念为“平台+资源”,即通过发布虚拟实验中心门户网站,建设开放式虚拟仿真实验教学管理和共享平台,并统一管理虚拟实验课程资源,从而开展各学科相关课程的虚拟实验教学。

\subsubsection{平台的主要功能}
开放式虚拟仿真实验教学的管理和共享平台包括以下主要子系统:

\begin{itemize}
    \item \textbf{虚拟实验中心门户网站:} 动态Web系统,内容包括中心介绍、实验教学、实验队伍、管理模式、设备与环境、教学特色、新闻公告等。
    \item \textbf{实验教务管理:} 包括课程库、培养计划、排课、选课、开课审核等功能。
    \item \textbf{实验教学管理:} 涵盖现场实验安排、虚拟实验安排、实验批改、考勤管理、成绩管理、实验报告等。
    \item \textbf{实验前理论学习:} 学生通过练习、自测、课件等方式学习实验理论知识。
    \item \textbf{实验过程智能指导:} 学生在实验过程中遇到问题时,系统提供指导信息。
    \item \textbf{实验结果自动批改:} 系统自动评判学生提交的实验结果,并给出分数和评分点。
    \item \textbf{数字化资源管理:} 上传和发布各种虚拟实验、仿真软件和演示动画。
    \item \textbf{实验室开放预约管理:} 管理实验室设备借出、实验室预约、实验预约、工位预约等。
    \item \textbf{师生互动交流:} 提供实时答疑、在线留言等功能。
    \item \textbf{系统管理:} 包括用户、分组、角色、权限、日志、备份管理和实时监控等。
\end{itemize}

\subsubsection{平台特点}
\begin{itemize}
    \item 系统根据学校实验教学整体需求设计,满足校级实验教学的需要;
    \item 可集成所有符合标准的第三方虚拟实验系统和软件;
    \item 经典实验采用B/S架构,便于学生使用和系统部署;
    \item 全面开放实验室资源,提供开放式实验教学服务,提升学生参与的主动性;
    \item 多角色应用体系和业务权限配置,满足不同用户的需求;
    \item 人性化的协同学习,支持在线或离线交流;
    \item 虚拟实验与现实实验相结合,丰富实验教学方式;
    \item 实验教学排课灵活,可统一安排或学生自选;
    \item 多种数据导入导出功能,方便数据汇总和统计;
    \item 可无缝集成到学校的教学管理系统中;
    \item 支持在线提交实验报告,并提供批注和智能批改功能;
    \item 根据现实实验室进行工位布局,方便预约和入座;
    \item 支持在线多媒体编辑宣传内容,实现实验中心与下设实验室门户网站的统一管理;
    \item 系统注重实验教学效果。
\end{itemize}

\subsection{北京润尼尔科技---VR教育应用硬件解决方案}

\subsubsection{方案背景}
传统的虚拟仿真实验室主要建立在机房内,通过PC机操作虚拟仿真软件,缺乏沉浸感,特别是在需要临场感的学科(如机械、心理学、艺术、医学等)的实验实训中。这些学科的实践教学通常受到实验危险系数高、设备昂贵、时空限制等因素的影响。紧跟虚拟现实技术的发展趋势,结合高校、职业院校和企业的培训需求,建设沉浸感强、交互丰富的虚拟仿真实验实训环境至关重要。

\subsubsection{建设思路}
VR教育应用硬件系列产品整合了虚拟现实(VR)、增强现实(AR)、混合现实(MR)、扩展现实(XR)、物联网(IOT)、大数据(DB)、人工智能(AI)等尖端技术,提供全方位的服务于实践教学、科学研究与展示的“124”立体化虚拟现实教育应用解决方案。方案包括一个中心、两个基地和四个实验室,通过三维虚拟环境中的操作交互,模拟和还原真实实验操作,提升实验中心形象,激发使用者学习兴趣。

\subsubsection{解决方案}
\begin{itemize}
    \item \textbf{虚拟现实创新教学实验室:} 提供大视野的立体化教学环境,通过数字建模技术还原现实场景,教师使用VR交互设备进行互动教学,学生佩戴眼镜进行体验。适用于公共安全灾害应急、大型工程施工现场模拟、复杂设备内部立体显示、虚拟工厂实训等。

    \item \textbf{全息教学实验室:} 由多套桌面级VR交互系统组成,全息成像技术将抽象知识以生动直观的方式呈现。适用于大型贵重设备的展示和微观世界原理的认知,增强学习者的理解和记忆。

    \item \textbf{沉浸式VR教学实验室:} 由VR头戴设备和图形工作站组成,提供完全虚拟的环境,通过视觉、听觉、触觉等全方位的体验,增强学习专注度和兴趣。

    \item \textbf{创新课程研发实验室:} 提供开放式研发平台,配备高端图形工作站、VR设备、数据手套等,采用校企合作方式提升学生的VR课程开发能力。
\end{itemize}

\subsubsection{方案特点}
\begin{itemize}
    \item \textbf{可灵活定制设计:} 根据客户需求进行专业设计,灵活配置软件和硬件。
    \item \textbf{开放性:} 硬件系统基于开放设计,适用于主流虚拟现实开发平台。
    \item \textbf{可扩展性:} 可在现有硬件环境中添加各类虚拟现实交互设备。
    \item \textbf{整体性:} 提供软硬件一体化的VR虚拟仿真实验实训环境解决方案。
\end{itemize}

