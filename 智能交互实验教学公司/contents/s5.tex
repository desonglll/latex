\subsection{联科集团(中国)有限公司}


\subsubsection{项目名称}

智能交互式实验教学平台研究与应用

\subsubsection{项目内容}

该项目旨在开发一个综合性的智能交互实验教学平台,通过现代技术提升教育实验的互动性、智能化和效果。项目主要包括以下几个方面:

\subsubsubsection{平台设计与开发}


设计并开发一个集成虚拟现实(VR)、增强现实(AR)和实时数据交互的实验教学平台。平台将支持多种实验场景,提供模拟实验环境和实时反馈,以增强学生的实验体验和学习效果。

\subsubsubsection{实验内容与课程开发}


与教育机构合作,基于平台开发针对不同学科和年级的实验课程。课程内容将包括科学实验、工程应用等领域,结合实际需求设计互动性强的实验任务。

\subsubsubsection{数据分析与反馈}


平台将集成数据分析工具,实时收集和分析学生在实验中的表现和反馈,以帮助教师评估学生的学习情况,并根据数据调整教学策略。

\subsection{讯飞幻境(北京)科技有限公司}

\subsubsection{项目名称}

\subsubsubsection{魔方AR智能课桌项目}

\subsubsubsection{AIR GURU智慧光AR学伴项目}

\subsubsubsection{AR智慧爱心创新教室项目}

\subsubsubsection{5G+智慧教育应用—实验教室及科普角项目}


\subsubsection{项目内容}

\subsubsubsection{魔方AR智能课桌项目}

该项目旨在通过AR(增强现实)技术,将实验课程与人工智能交互桌面融为一体,实现知识内容的全场景展示与交互。课桌集安全、低成本、物联网、自动评测等特点于一身,能够打破传统素质教育现状,提升教学效果。适用于中小学科学实验课程,通过增强现实技术将实物卡牌与虚拟情景进行数字化的叠加,模拟多种实验场景,增强学生的实践能力和学习兴趣。

\subsubsubsection{AIR GURU智慧光AR学伴项目}

该项目推出了一款陪伴式的智能早教机器人,集七大功能于一体,包括安卓智能主机、WIFI、蓝牙音箱、AR Camera、AR互动学习、学习护眼灯、安全小夜灯等。通过寓教于乐的方式,提供亲子互动体验,助力儿童早期教育。适用于家庭早教和幼儿园教育,为儿童提供丰富的学习资源和互动体验。

\subsubsubsection{AR智慧爱心创新教室项目}

该项目是讯飞幻境针对乡村学校推出的公益项目,旨在通过AR技术赋能乡村教育。项目包括AR智慧爱心创新教室、AR智慧爱心探索教室、AR智慧爱心科普角等多种应用场景,为偏远地区学校的孩子们提供现代化的教学环境和资源。适用于小学、初中阶段的科学科普课程、物化生实验等。通过AR技术模拟真实教学环境,安全进行各类实验教学,提高学生学习效率,满足个性化学习需求。

\subsubsubsection{5G+智慧教育应用—实验教室及科普角项目}

该项目是讯飞幻境联合运营商与青岛市西海岸新区共同开展的智慧教育应用项目。通过5G、AI、AR、XR等新一代新兴关键技术,降低实验室建设成本,提高实验教学安全性,优化实验室跨学科应用。主要应用于中学理化生实验教学、小学科普教学、传统实验室的升级改造等。通过幻境AR智慧实验课桌套组、数字课程等核心产品,实现实验教学的全流程数字化、轻实操和多模态教学。

\subsubsection{相关技术}

\subsubsubsection{增强现实(AR)技术}

AR技术是一种将虚拟信息叠加到现实世界中的技术,通过摄像头捕捉现实场景并与计算机生成的虚拟信息结合,实现用户与虚拟对象的实时交互。在魔方AR智能课桌和AIR GURU智慧光AR学伴项目中,AR技术被用于实现实物卡牌与虚拟情景的数字化叠加,提供直观、生动的学习体验。

\subsubsubsection{虚拟现实(VR)技术}

VR技术是一种利用计算机模拟出一个三维环境的技术,用户通过佩戴VR设备进入该环境,与虚拟世界进行交互。在5G+智慧教育应用项目中,VR技术被用于构建逼真的教学场景,如火星探测、细胞世界等,为学生提供沉浸式的学习体验。
\subsubsubsection{人工智能(AI)技术}

AI技术是一种模拟人类智能的技术,包括机器学习、自然语言处理、计算机视觉等多个领域。在讯飞幻境的多个项目中,AI技术被用于实现智能评测、自动答疑、个性化推荐等功能,提升教学效果和学习体验。
\subsubsubsection{5G通信技术}

5G是一种高速、低延迟的通信技术,能够支持大规模的设备连接和高速数据传输。在5G+智慧教育应用项目中,5G通信技术被用于实现远程教学、实时互动等功能,提升教学资源的共享和利用效率。

\subsection{广州视睿电子科技有限公司}

\subsubsection{项目名称}

\subsubsubsection{智慧教室建设项目}

\subsubsubsection{希沃互动课堂项目}


\subsubsection{项目内容}

\subsubsubsection{智慧教室建设项目}

该项目旨在通过集成希沃的硬件设备和软件平台,打造智慧化的实验教学环境。具体内容包括安装希沃交互智能平板、智慧黑板等硬件设备,以及部署希沃白板、授课助手等软件工具,实现实验教学的数字化、互动化和智能化。适用于各类学校的实验室、实训室等教学场所,通过智能交互技术提升实验教学的效果和体验。
\subsubsubsection{希沃互动课堂项目}

该项目基于希沃的软硬件组合,为实验教学提供互动、即时测评、数据统计等功能。通过希沃白板5、授课助手等工具,教师可以轻松制作实验课件,实现与学生的实时互动和反馈。广泛应用于物理、化学、生物等学科的实验教学,以及各类需要互动和反馈的教学场景。

\subsubsection{相关技术}

\subsubsubsection{智能交互技术}

智能交互技术是实现智慧教室和互动课堂的核心。希沃的交互智能平板和智慧黑板采用高精度触控技术,支持多点触控和手势操作,为师生提供流畅的交互体验。同时,通过集成摄像头、麦克风等硬件设备,实现远程互动和资源共享。在实验教学中,教师可以通过触控屏幕展示实验步骤和结果,学生可以通过手势或触控笔进行互动操作,提高实验的参与度和理解度。
\subsubsubsection{大数据分析技术}

大数据分析技术用于收集和分析学生在学习过程中的数据,为教师提供精准的教学反馈和学情分析。希沃的软件平台支持教学小数据的静默采集与可视化呈现,帮助教师了解学生的学习情况,优化教学策略。在实验教学中,教师可以通过希沃的软件平台收集学生的实验数据,进行统计分析,了解学生在实验过程中的难点和错误点,从而进行有针对性的指导和辅导。
\subsubsubsection{云计算与物联网技术}

云计算和物联网技术为智慧教室和互动课堂提供强大的技术支撑。通过云计算平台,实现教学资源的共享和远程互动;通过物联网技术,实现教学设备的智能化管理和控制。在实验教学中,教师可以通过云平台访问和共享实验资源,实现跨地域的远程教学和协作;同时,物联网技术可以实现对实验设备的远程监控和管理,提高设备的使用效率和安全性。

\subsection{认知智能国家重点实验室}

\subsubsection{项目名称}

认知智能国家重点实验室在智能交互实验教学关键技术研究与应用

\subsubsection{项目内容}

\subsubsubsection{项目背景与目的}

该项目旨在利用认知智能技术提升实验教学的效果和质量,特别是在职业教育和基础教育中。通过虚拟现实(VR)、增强现实(AR)、人工智能(AI)等技术的应用,实现实验教学的智能化、互动化和个性化。认知智能国家重点实验室通过整合虚拟现实、增强现实、人工智能等先进技术,旨在构建一个智能化、互动化、个性化的实验教学平台,为职业教育和基础教育提供全新的教学方式和体验。同时,通过智能评估与反馈系统,实现对学生学习效果的实时监控和指导,提升实验教学的质量和效果。
\subsubsubsection{主要研究方向}

虚拟现实与增强现实的融合应用:开发虚拟实验室和增强现实实验平台,模拟真实实验环境,提供沉浸式学习体验。
智能交互技术:利用自然语言处理(NLP)、语音识别、手势识别等技术,实现人机自然交互。
认知模型与智能评估:构建学生认知模型,通过大数据分析和机器学习技术,对学生的学习行为和学习效果进行智能评估与反馈。
实验教学资源共享与协作:开发基于云计算的实验教学资源共享平台,实现教学资源的高效管理与协作共享。
\subsubsubsection{预期成果}

智能交互实验教学平台:建立一个集成VR、AR、AI等技术的综合性实验教学平台,支持多种学科的实验教学。
个性化学习与评估系统:开发基于认知模型的个性化学习与评估系统,提供针对每个学生的个性化学习路径与反馈建议。
教学资源库与教师培训:建设丰富的实验教学资源库,并开展教师培训,提升教师在智能实验教学中的应用能力。
\subsubsection{相关技术}
\subsubsubsection{虚拟现实(VR)与增强现实(AR)技术}

VR技术:利用头戴显示设备和计算机生成的虚拟环境,模拟真实实验场景,提供沉浸式学习体验。
AR技术:通过智能手机或平板设备,将虚拟信息叠加到现实环境中,增强学生的实验操作感知。
\subsubsubsection{自然语言处理(NLP)与语音识别技术}

NLP技术:用于理解和处理学生的自然语言输入,实现人机对话和智能问答。
语音识别技术:将学生的语音输入转换为文本,实现语音控制和语音交互。
\subsubsubsection{机器学习与大数据分析}

机器学习:用于构建学生的认知模型,分析学习行为数据,提供个性化学习建议。
大数据分析:收集和分析大量的实验教学数据,发现教学中的问题和优化方向。
\subsubsubsection{智能评估与反馈技术}

智能评估:基于学生的学习数据,自动评估学生的学习效果和掌握情况。
反馈技术:提供即时的学习反馈和指导,帮助学生及时纠正错误,优化学习路径。

\subsection{北京竞业达}

\subsubsection{项目名称}

智能交互实验教学关键技术研究与应用示范

\subsubsection{项目内容}

\subsubsubsection{智能实验教学平台建设}

通过整合虚拟现实(VR)、增强现实(AR)和混合现实(MR)技术,构建一个智能实验教学平台,使学生能够在虚拟环境中进行实验操作,从而提高学习效率和实验效果。
\subsubsubsection{实时交互与反馈机制}

开发实时交互系统,使教师能够实时监控学生的实验进展,并提供即时反馈,帮助学生更好地理解实验内容。

个性化学习路径:利用人工智能技术,根据学生的学习情况和进度,制定个性化的学习路径,确保每个学生都能得到最适合自己的学习资源和指导。
\subsubsubsection{数据分析与评估系统}

建立全面的数据采集和分析系统,对学生的实验过程进行详细记录和分析,为教师提供科学的评估依据。

\subsubsection{相关技术}

\subsubsubsection{虚拟现实(VR)技术}

通过高质量的虚拟现实设备和软件,学生可以在逼真的虚拟环境中进行实验操作,获得身临其境的体验。
\subsubsubsection{增强现实(AR)技术:}

结合增强现实技术,将虚拟信息与现实环境相结合,帮助学生更直观地理解复杂的实验概念。
\subsubsubsection{混合现实(MR)技术}

利用混合现实技术,学生可以在真实世界中进行虚拟实验操作,增强学习的互动性和参与感。
\subsubsubsection{人工智能(AI)技术}

应用机器学习和自然语言处理技术,开发智能辅导系统,为学生提供个性化的学习建议和指导。
\subsubsubsection{大数据分析技术}

通过对学生实验数据的收集和分析,提供科学的教学评估和改进建议,优化教学过程。

北京竞业达在这些技术的应用和整合上,致力于构建一个高效、智能、互动的实验教学平台,提升教育质量和学生的学习体验。

\subsection{亮风台与河北师范大学}

\subsubsection{项目名称}

智能交互实验教学关键技术研究与应用

\subsubsection{项目内容}

该项目的主要目标是利用先进的智能交互技术来提升实验教学的质量和效果,尤其是在职业教育和基础教育中的应用。
\subsubsubsection{虚拟现实(VR)技术的应用}

开发基于VR的实验教学平台,使学生能够在虚拟环境中进行实验操作,增强学习的互动性和沉浸感。
\subsubsubsection{实时互动技术的应用}

利用实时互动技术,使教师和学生能够在实验过程中进行实时的沟通和反馈,提高教学的即时性和有效性。
\subsubsubsection{启发式推理技术的应用}

通过启发式推理技术,提供智能化的教学辅助,帮助学生更好地理解实验过程中的关键概念和原理。
\subsubsubsection{智能监测与评估系统}

建立智能化的监测与评估系统,实时跟踪学生的学习进度和实验操作,提供个性化的教学建议和反馈。

\subsubsection{相关技术}

\subsubsubsection{增强现实(AR)与虚拟现实(VR)技术}

利用AR和VR技术构建虚拟实验室,使学生能够在虚拟环境中进行实验操作,减少实际实验中的成本和风险。
\subsubsubsection{人工智能(AI)技术}

通过机器学习和自然语言处理技术,开发智能化的教学辅助系统,实现对学生学习行为的智能分析和反馈。
\subsubsubsection{物联网(IoT)技术}

将IoT技术应用于实验设备的连接和数据采集,实现对实验过程的实时监控和数据分析。
\subsubsubsection{云计算技术}

利用云计算平台,提供实验教学的云端服务,实现数据的存储、处理和共享,提高系统的扩展性和灵活性。
\subsubsubsection{大数据技术}

通过大数据技术对实验教学数据进行分析,发现学生的学习规律和潜在问题,提供数据驱动的教学优化方案。

通过这些技术的综合应用,亮风台与河北师范大学的合作项目旨在打造一个智能化、互动性强、效果显著的实验教学平台,提升学生的实验技能和综合素质。

\subsection{北京华晟经世信息技术股份有限公司}

\subsubsection{项目名称}

智能虚拟实验室平台

\subsubsection{项目内容}

该项目致力于构建一个智能虚拟实验室平台,旨在为教育机构提供一个高效、互动性强的实验教学环境。平台的主要功能包括:
\subsubsubsection{虚拟实验环境}

通过虚拟现实(VR)技术创建高度还原的实验环境,学生可以在虚拟实验室中进行各种实验操作,而无需实际实验器材。
\subsubsubsection{实时互动}

平台集成了实时互动功能,允许教师和学生在实验过程中进行实时交流,教师可以远程指导学生完成实验操作,学生之间也可以协作完成实验任务。
\subsubsubsection{数据记录与分析}

系统自动记录实验数据,并提供数据分析工具,帮助学生理解实验结果,教师也可以通过分析数据评估学生的实验技能和理解能力。
\subsubsubsection{模拟实验}

除了传统实验,平台还提供了各种模拟实验,帮助学生进行风险较大的实验操作,降低实验过程中的安全风险。

\subsubsection{相关技术}

\subsubsubsection{虚拟现实(VR)技术}

使用VR技术创建沉浸式的实验环境,使学生可以在虚拟空间中进行实验操作,模拟真实的实验情境,增强学习的沉浸感和真实感。
\subsubsubsection{实时数据处理与交互技术}

平台利用实时数据处理技术确保实验过程中的数据即时更新,并通过实时交互技术实现师生之间的即时沟通与指导。
\subsubsubsection{人工智能(AI)技术}

集成AI技术进行智能分析和建议,AI可以根据学生的实验操作提供实时反馈和优化建议,帮助学生提高实验操作的准确性和效率。
\subsubsubsection{云计算与大数据分析}

通过云计算技术支持平台的高效运转,同时利用大数据分析技术对实验数据进行深度分析,为教育决策提供数据支持。
\subsubsubsection{多模态交互技术}

平台支持多种交互方式,如语音识别、手势控制等,提高用户体验和交互便捷性。
\subsubsubsection{应用场景}

基础教育:适用于中小学实验教学,帮助学生在安全的虚拟环境中进行各种科学实验。

职业教育:为职业学校提供实操训练的平台,增强学生的实际操作能力。

高等教育:支持大学的实验课程,提供更为复杂和高端的实验环境,促进学术研究和技术创新。

这个项目旨在通过智能化的技术手段提升实验教学的效率和效果,帮助学生更好地掌握实验技能,同时也为教师提供了更多的教学工具和手段。
