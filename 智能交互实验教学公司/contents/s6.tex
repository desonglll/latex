
\subsection{清华大学与交互未来(北京)科技有限公司}


\subsubsection{项目名称}

\subsubsubsection{智能交互实验教学平台开发}
\subsubsubsection{智能交互实验教学课程与教材开发}

\subsubsection{项目内容}

1. 开发一套集教学、实验、研发于一体的智能交互实验教学平台,该平台将融合清华大学在智能人机交互、人因工程、认知科学等领域的最新研究成果,为学生提供丰富的实践机会和前沿技术体验。

2. 通过该平台,学生可以参与到真实的智能交互系统设计与实现过程中,提升解决实际问题的能力。同时,该平台也可作为展示清华大学智能交互研究成果的重要窗口,吸引更多企业和研究机构进行合作与交流。

3. 开发一系列具有创新性和实用性的实验教学课程和教材,这些课程和教材将涵盖智能交互技术的基本原理、实验方法、案例分析等内容,为培养高素质的智能交互人才提供有力支持。

4. 通过推广使用这些课程和教材,提升高校智能交互实验教学的整体水平和质量。同时,也可以为相关企业和研究机构提供培训服务,推动智能交互技术在更广泛范围内的普及和应用。

\subsubsection{相关技术}

\subsubsubsection{自然语言处理技术}
用于实现人机交互中的语音识别、语义理解和文本生成等功能,提升交互的自然性和准确性。
\subsubsubsection{计算机视觉技术}
通过图像识别、目标跟踪、场景理解等手段,实现人机交互中的视觉感知和交互,为用户提供直观、便捷的交互体验。
\subsubsubsection{情感计算技术}
通过分析用户的面部表情、声音语调等情感信号,理解用户的情感状态和需求,为用户提供更加个性化、情感化的交互体验。
\subsubsubsection{多模态交互技术}
结合语音、手势、眼动等多种交互模态,实现更加自然、高效的人机交互方式,提升用户的使用体验和满意度。
\subsubsubsection{虚拟现实/增强现实技术}
通过构建虚拟或增强的交互环境,为用户提供沉浸式的交互体验,特别适用于实验教学等场景。
\subsubsubsection{人工智能技术}
包括机器学习、深度学习等,用于提升智能交互系统的智能水平,使其能够更好地理解用户需求、优化交互流程、提供个性化服务等。


\subsection{讯飞幻境(北京)科技有限公司}


\subsubsection{项目名称}

\subsubsubsection{幻境AR智能课桌}
\subsubsubsection{幻境AI黑板与AI智屏}

\subsubsection{项目内容}

1. 幻境AR智能课桌是讯飞幻境的核心产品之一,集安全、低成本、物联网、自动评测等特点于一体。通过将实验课程与人工智能交互桌面融为一体,实现知识内容的全场景展示与交互。

2. 利用增强现实技术将实物卡牌与虚拟情景进行数字化的叠加,模拟实验场景,使学生在无安全风险的情况下进行实验操作。同时,课桌内部植入功能强大的电脑,支持多种教学内容的3D图形化展现。

3. 结合人工智能技术的黑板和智屏产品,支持智能书写、语音交互、内容推送等功能。

4. 利用语音识别和自然语言处理技术实现人机对话,支持教师快速生成教案和课堂笔记。同时,通过内容推送功能将教学资源实时传输给学生,提高教学效率。

\subsubsection{相关技术}

\subsubsubsection{全沉浸可视化技术}
结合人工智能与虚拟现实技术,实现高度沉浸式的教学体验。
\subsubsubsection{语音识别+全息投影技术}
通过语音识别技术实现人机交互,结合全息投影技术展示三维教学内容。
\subsubsubsection{图像识别+空间定位技术}
精准识别教学对象并进行空间定位,为实验操作提供精准指导。
\subsubsubsection{云备课技术+多终端管控技术}
支持教师远程备课和课堂管理,提高教学效率。


\subsection{国开泛在(北京)教育科技有限公司}


\subsubsection{项目名称}

\subsubsubsection{泛在智能问答机器人云平台}
\subsubsubsection{智能交互黑板}
\subsubsubsection{思政 VR 实训室}
\subsubsubsection{实验云平台}

\subsubsection{项目内容}


1. 泛在智能问答机器人云平台是一个基于自然语言理解的语义检索、多渠道知识服务、大规模知识库构建的平台,专为教育机构提供智能问答解决方案。该平台通过SaaS模式,为教育机构提供招生与教务方面的智能问答服务,有效解决高峰期师生咨询量大、传统咨询方式教师压力大、咨询问题难以积累与总结等问题。

2. 智能交互黑板是新一代多媒体教学平台,集触控、高清大屏显示、电脑主机、普通黑板等诸多功能于一身。它涵盖了粉笔书写、多人触控、多媒体教学的功能,利用全贴合工艺及触控技术,实现传统教学黑板和可感知互动黑板之间的无缝对接,使教师更易教学,学生健康学习。

3. 针对思政课教学存在的灌输式、体验感差、实践教学形式单一、教学内容滞后等问题,国开泛在推出了思政VR实训室方案。该方案包括由数百种数字化思政教学资源与多种信息化教学设备组成的红色文化主题馆、智慧教室和VR实践教学系统,旨在通过沉浸式、互动式的学习体验,提高思政教学的效果和质量。

4. 实验云平台通常是一个集实验资源管理、实验预约、实验数据记录与分析、实验报告提交等功能于一体的综合性平台。它旨在提高实验教学的效率和管理水平,促进实验资源的共享和利用。

\subsubsection{相关技术}

\subsubsubsection{自然语言理解}
用于解析用户输入的自然语言问题,理解其意图和上下文。
\subsubsubsection{语义检索}
在理解用户问题后,在大规模知识库中进行高效检索,找到最匹配的答案。
\subsubsubsection{知识图谱}
以结构化的方式存储知识,便于机器人进行推理和查询,提供更深入、准确的答案。
\subsubsubsection{机器学习}
通过历史问答记录不断优化机器人的性能,提供个性化的回答。
\subsubsubsection{多渠道接入技术}
支持网页、微信公众号、APP等多种接入方式,满足不同场景下的智能问答需求。
\subsubsubsection{触控技术}
实现多人同时触控操作,提高课堂教学的互动性。
\subsubsubsection{高清大屏显示技术}
提供清晰、生动的视觉效果,增强教学内容的吸引力。
\subsubsubsection{全贴合工艺}
减少反光和阴影,保护学生视力。
\subsubsubsection{电脑主机集成}
内置电脑主机,支持多媒体教学资源的直接调用和展示。
\subsubsubsection{虚拟现实技术}
构建高度还原的虚拟环境,提供沉浸式的学习体验。
\subsubsubsection{增强现实技术}
辅助VR技术,实现虚实结合的互动教学。
\subsubsubsection{多媒体教学资源开发技术}
制作丰富的数字化思政教学资源,满足教学需求。
\subsubsubsection{智能督导系统}
监控教学过程,确保教学质量。


\subsection{南京恒点信息技术有限公司}


\subsubsection{项目名称}

\subsubsection{恒点 MOOL 云}
\subsubsection{虚拟仿真行业视角}
\subsubsection{虚拟仿真实验项目开发软件}

\subsubsection{项目内容}

1. 恒点MOOL云是南京恒点信息技术有限公司推出的一种大规模在线开放实验室(Massive Open Online Labs)平台,旨在通过互联网技术打破传统实验室的时空限制,为学生提供更加丰富、便捷的实验学习体验。该平台不仅支持理论知识的学习,更注重实践动手能力的培养,通过虚拟仿真技术让学生在虚拟环境中进行实验操作,提高教学效果和学生的学习兴趣。

2. 提供虚拟仿真教学资源共享平台,为教育机构提供丰富的虚拟仿真实验课程和资源;开发定制化的虚拟仿真解决方案,满足不同领域和行业的特定需求;开展虚拟仿真技术的培训和咨询服务,提升用户的技术水平和应用能力。

3. VRC-Editor是南京恒点信息技术有限公司推出的一款虚拟仿真实验项目开发软件,它采用零编程、流程化、模块化的设计理念,让非专业开发人员也能轻松创建高质量的虚拟仿真实验项目。该软件支持多种终端设备和平台,能够广泛应用于教育、科研、培训等领域。

\subsubsection{相关技术}

\subsubsubsection{云计算技术}
提供强大的数据存储和计算能力,支持大规模用户并发访问。
\subsubsubsection{虚拟仿真技术}
构建高度还原的虚拟实验环境,让学生在虚拟环境中进行实验操作,体验真实的实验过程。
\subsubsubsection{大数据分析}
记录并分析学生的学习数据,为教师提供教学反馈,优化教学内容和方法。
\subsubsubsection{交互式设计}
提供用户友好的界面和交互方式,方便学生进行实验操作和学习交流。
\subsubsubsection{三维建模与渲染技术}
构建逼真的虚拟场景和物体。
\subsubsubsection{物理引擎技术}
模拟真实的物理现象和行为。
\subsubsubsection{跨平台兼容技术}
确保虚拟仿真内容能够在不同设备和平台上流畅运行。
\subsubsubsection{零编程设计理念}
通过图形化界面和拖拽式操作,降低开发门槛,提高开发效率。
\subsubsubsection{模块化构建}
提供丰富的模块和组件库,用户可以根据需求自由组合和定制。
\subsubsubsection{多人协同编辑}
支持多人在线协同工作,提升团队开发效率。


\subsection{上海商汤智能科技有限公司}


\subsubsection{项目名称}

\subsubsubsection{智能交互实验平台开发}
\subsubsubsection{AI实验课程和教材开发}
\subsubsubsection{产学研合作项目}
\subsubsubsection{智能交互实训系统}

\subsubsection{项目内容}

1.商汤科技可能开发了基于人工智能技术的智能交互实验平台,该平台集成了多种智能交互技术和实验资源,支持学生进行自主实验和探索。

2.结合商汤科技在人工智能领域的研究成果,公司可能开发了一系列针对高中和中小学的AI实验课程和教材,涵盖从基础概念到高级应用的多个层次。

3.商汤科技与国内多所高校和科研机构合作,共同开展智能交互实验教学的研究与应用示范项目,推动人工智能技术在教育领域的应用和发展。

4.针对特定行业或领域的需求,商汤科技可能开发了智能交互实训系统,如智慧医疗、智慧交通等领域的实训系统,帮助学生掌握实际应用中的智能交互技术。

\subsubsection{相关技术}

\subsubsubsection{人工智能核心技术}
包括感知智能、决策智能、智能内容生成和智能内容增强等关键技术领域,这些技术为智能交互实验平台提供了强大的支撑。
\subsubsubsection{虚拟仿真技术}
通过构建高度还原的虚拟实验环境,学生可以在虚拟环境中进行实验操作和学习,提高教学效果和学习体验。
\subsubsubsection{交互式设计技术}
提供直观、易用的交互界面和操作方式,使得学生和教师能够轻松上手并高效使用智能交互实验平台。
\subsubsubsection{大数据分析技术}
记录并分析学生的学习数据和行为数据,为教师提供教学反馈和优化建议,进一步提升教学效果。
\subsubsubsection{AI芯片与传感器技术}
商汤科技在AI芯片和传感器方面的积累也为智能交互实验教学提供了技术支持,使得实验平台在数据处理和交互响应方面更加高效和准确。
\subsubsubsection{原创深度学习平台SenseParrots}
商汤科技自主研发的深度学习平台SenseParrots对超深的网络规模、超大的数据学习以及复杂关联应用等支持更具优势,为智能交互实验平台提供了强大的算法支持。


\subsection{中国教科院职教所}


\subsubsection{项目名称}

\subsubsubsection{智能交互实验平台开发}
\subsubsubsection{智能教学资源库建设}

\subsubsection{项目内容}

1.开发集成了多种智能交互技术的实验平台,支持学生进行自主探究和协作学习。
平台可能包含虚拟仿真实验环境、智能数据分析工具、交互式学习界面等模块。

2.构建涵盖多个学科和领域的智能教学资源库,包括实验案例、教学视频、虚拟仿真模型等。
资源库支持在线访问和下载,方便教师和学生随时获取所需资源。

\subsubsection{相关技术}

\subsubsubsection{智能交互技术}
包括触控技术、语音识别技术、手势识别技术等,这些技术使得实验设备和教学软件能够与用户进行自然、直观的交互。
\subsubsubsection{虚拟仿真技术}
通过构建虚拟实验环境,模拟真实或理想化的实验场景,使学生在安全、可控的环境中进行实验操作和学习。
\subsubsubsection{大数据分析技术}
对实验过程中产生的数据进行收集、分析和挖掘,为教师提供教学反馈和学生学习行为分析,帮助教师优化教学内容和方法。
\subsubsubsection{人工智能算法}
利用机器学习、深度学习等算法对实验数据进行智能处理和分析,提高实验结果的准确性和可靠性。
\subsubsubsection{云计算与边缘计算技术}
提供强大的数据存储和计算能力,支持大规模用户并发访问和实时数据处理。
