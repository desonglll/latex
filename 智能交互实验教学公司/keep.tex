
\subsection{中国科学院大学与北京光年无限科技有限公司}\label{中国科学院大学与北京光年无限科技有限公司}

\subsubsection{项目名称}

基于机器学习的人机智能交互设备关键技术与应用

\subsubsection{项目内容}

\subsubsubsection{知识深度融合与知识聚合技术}

该项目围绕人工智能交互设备技术与应用中的难点,将机器学习与最优化理论相结合,首次创建了智能知识感知理论框架,涵盖了自然语言理解、机器视觉、语音信号处理等多个领域,为多模态人机交互研究打下理论基础。通过构建基于知识图谱的典型工程实验知识库,实现了实验过程与知识的深度融合与高效聚合。

\subsubsubsection{虚拟实验操作系统}

项目研发了实验现象和实验设备虚实互联、即时交互、裸眼可视的虚拟实验操作系统,利用虚拟现实技术模拟真实实验环境,提升学生的学习体验和实验效率。

\subsubsubsection{实验步骤生成与引导技术}

基于场景化感知和启发式推理,项目研究了实验步骤的自动生成与智能引导技术,帮助学生更高效地完成实验过程,减少操作错误。
多传感器融合技术

在实验过程跟踪与分析方面,项目运用了声、光、电、气体、视觉等多传感器融合技术,实现对实验过程的全方位、高精度监测,为实验结果的准确性和可靠性提供保障。

\subsubsubsection{平台研发与应用示范}

构建了典型工程科技实验虚实互联智能学习平台,并通过广泛的应用示范展示了其在教学、科研及产业应用中的巨大潜力。

\subsubsection{相关技术}

\subsubsubsection{机器学习与人机交互融合}


该项目将机器学习与人机交互技术紧密结合,旨在提升智能终端设备的智能化水平。机器学习技术使得设备能够从大量数据中自动学习和改进,而人机交互技术则关注于如何设计有效的交互界面和方式,以满足用户的需求和期望。

\subsubsubsection{智能知识感知理论框架}


项目首次创建了智能知识感知理论框架,这一框架涵盖了自然语言理解、机器视觉、语音信号处理等多个领域,为多模态人机交互研究奠定了理论基础。这一框架不仅增强了模型的多模态互补推理能力,还提升了算法效率,优化了用户体验。

\subsubsubsection{多模态数据表示与处理方法}


针对多模态数据表示差异问题,项目提出了一种全新的知识认知与表示方法。这种方法能够有效地处理来自不同模态的数据,实现数据的高效融合和互补,从而提高了交互的准确性和自然度。

\subsubsubsection{大语言模型构建}


项目构建了多模态复杂场景下的大语言模型,在多模态表示、学习、映射、对齐与融合等领域取得了重要突破。这一模型不仅具有强大的语义理解能力,还能够处理复杂场景下的多模态交互任务,为用户提供更加智能化的服务。

\subsubsubsection{自主知识产权的智能硬件生态体系}


通过创新性技术及产品的研究,项目将自主研发的操作系统与国产芯片相结合,打破了智能教育领域的国际垄断格局。这一举措不仅提升了我国在该领域的自主创新能力,还推动了相关产业的发展和升级。

\subsubsubsection{个性化服务与智能推荐}

基于机器学习的人机交互技术,智能终端设备能够学习用户的使用习惯和喜好,并根据这些信息提供个性化的服务。例如,通过分析用户的上网记录和应用使用情况,设备可以预测用户可能感兴趣的内容,并在未来的搜索结果中优先显示这些内容。这种个性化的推荐系统大大提高了用户的满意度和使用体验。

\subsubsubsection{语音与视觉交互技术}


项目在语音和视觉交互方面也取得了显著进展。通过机器学习,智能终端设备能够逐渐学习不同用户的语音特征和使用习惯,从而更好地识别和理解用户的指令。同时,设备还可以通过人脸识别和姿势识别等技术,实现更加智能化的用户认证和交互方式。

\subsubsubsection{图灵机器人平台}


北京光年无限科技有限公司的图灵机器人平台是该项目的重要成果之一。图灵机器人平台具备强大的中文语义理解能力,对中文语义的理解准确率高达90\%以上。该平台可为智能化软硬件产品提供中文语义分析、自然语言对话、深度问答等人工智能技术服务,广泛应用于机器人、智能家居、智能车载、智能客服等多个领域。

\subsection{哈尔滨工业大学智能实验系统}\label{哈尔滨工业大学智能实验系统}

\subsubsection{项目内容}

\subsubsubsection{智能实验指导系统}


该系统能够根据学生的实验需求和水平,提供个性化的实验指导和建议。通过自然语言处理和机器学习技术,系统能够理解学生的问题,并给出准确的解答和操作步骤。
实验数据自动分析系统

利用大数据分析和机器学习算法,该系统能够自动处理和分析实验数据,提取关键信息,并生成详细的实验报告。这不仅可以减轻学生的数据处理负担,还能提高数据分析的准确性和效率。

\subsubsubsection{虚拟实验助手}


结合虚拟现实(VR)和增强现实(AR)技术,开发虚拟实验助手,为学生提供沉浸式的实验体验。学生可以在虚拟环境中进行实验操作,观察实验现象,并实时获得反馈和指导。

\subsubsubsection{智能实验设备控制}


通过物联网和智能控制技术,实现实验设备的远程控制和自动化操作。教师可以通过智能实验助手系统,对实验设备进行远程监控和调控,确保实验过程的顺利进行

\subsubsection{项目技术}

\subsubsubsection{人工智能和机器学习}


利用深度学习、自然语言处理、计算机视觉等人工智能技术,实现智能实验助手的智能化和自动化。通过训练模型,使系统能够理解和处理复杂的实验数据和问题。

\subsubsubsection{大数据分析和处理}


采用大数据分析和处理技术,对实验数据进行快速处理和挖掘。通过提取关键信息,发现数据之间的关联和规律,为实验指导和决策提供支持。

\subsubsubsection{物联网和智能控制}


利用物联网技术,实现实验设备的互联互通和远程监控。通过智能控制算法,对实验设备进行自动化控制和调节,提高实验过程的稳定性和可靠性。

\subsubsubsection{虚拟现实和增强现实}


结合VR和AR技术,开发虚拟实验助手系统。通过创建逼真的虚拟实验环境,为学生提供沉浸式的实验体验,并实时提供反馈和指导。


\subsection{中科院生物物理研究所}\label{中科院生物物理研究所}

\subsubsection{项目内容}

\subsubsubsection{新型生物传感器开发}

中科院生物物理研究所可能会致力于开发新型生物传感器,这些传感器可能针对特定的生物分子、细胞或生理过程进行高灵敏度、高选择性的检测。

项目可能涉及传感器的设计、制备、优化以及性能评估等多个环节。

\subsubsubsection{生物传感器在医疗诊断中的应用}

研究所可能会研究生物传感器在医疗诊断中的潜在应用,如血糖监测、癌症早期检测、感染病原体筛查等。

项目可能旨在开发便携式、快速响应的生物传感器设备,以提高医疗诊断的效率和准确性。

\subsubsubsection{生物传感器在环境监测中的应用}

鉴于环境污染问题的日益严重,研究所可能会关注生物传感器在环境监测中的应用,如检测水质、空气质量以及土壤中的有害物质。

项目可能旨在开发能够实时监测和预警环境污染的生物传感器系统。

\subsubsubsection{生物传感器在农业和食品安全中的应用}

研究所还可能研究生物传感器在农业和食品安全领域的应用,如检测作物生长状况、土壤参数以及食品中的有害物质(如重金属、农药残留等)。

这些应用有助于提高农业生产的效率和食品的安全性。

\subsubsection{项目技术}

\subsubsubsection{生物识别元件的制备}

生物传感器通常包含生物识别元件(如酶、抗体、细胞等),这些元件的选择和制备对传感器的性能至关重要。

研究所可能会采用先进的生物技术(如基因工程、蛋白质工程等)来制备高性能的生物识别元件。

\subsubsubsection{传感器设计与优化}

传感器的设计需要考虑灵敏度、选择性、稳定性等多个因素。

研究所可能会采用微纳加工技术、材料科学等领域的最新成果来设计和优化传感器结构。

\subsubsubsection{信号处理与分析}

生物传感器产生的信号需要经过处理和分析才能得到有用的信息。

研究所可能会开发先进的信号处理算法和数据分析软件,以提高信号处理的精度和效率。

\subsubsubsection{跨学科合作}

生物传感器技术涉及生物学、化学、物理学、电子学等多个学科领域。

研究所可能会与其他领域的科研机构和企业开展跨学科合作,共同推动生物传感器技术的发展和应用。

\subsection{沈阳新松虚拟现实研究院}\label{沈阳新松虚拟现实研究院}


\subsubsection{项目名称}

\subsubsubsection{虚拟教室与虚拟实验室项目}

\subsubsubsection{AR(增强现实)教学系统项目}

\subsubsubsection{人机协同技术实验项目}

\subsubsubsection{云实习系统项目}


\subsubsection{项目内容}

\subsubsubsection{虚拟教室与虚拟实验室项目}

该项目旨在通过虚拟现实技术构建逼真的教学场景,模拟真实的教室和实验室环境。学生可以在虚拟环境中进行实验操作、课程学习和互动交流,体验沉浸式的学习环境。

\subsubsubsection{AR教学系统项目}

该项目开发了AR教学系统,将虚拟信息叠加到现实世界中,为学生提供直观、生动的学习体验。系统可应用于机械、医学等多个领域,帮助学生更好地理解复杂的三维结构和工作原理。

\subsubsubsection{人机协同技术实验项目}

该项目研究虚拟现实技术与人工智能、机器人等技术的结合,实现人与机器之间的无缝协作。在实验教学中,学生可以通过与虚拟机器人或智能系统的互动,学习人机协同的原理和应用。

\subsubsubsection{云实习系统项目}

该项目利用虚拟现实和云计算技术,将实习环境搬到云端。学生可以通过互联网接入云实习系统,在任何时间、任何地点进行实习操作和学习。

\subsubsection{相关技术}

\subsubsubsection{虚拟现实技术(VR)}

利用计算机图形学和仿真技术生成三维人工环境,用户通过传感器设备融入虚拟空间,以自然的方式与三维环境交互。在虚拟教室和虚拟实验室项目中,VR技术用于构建逼真的教学场景;在云实习系统中,VR技术提供沉浸式的实习体验。

\subsubsubsection{增强现实技术(AR)}

将计算机生成的图形、数据等虚拟对象叠加到真实场景上,用户可以通过多种方式与虚拟对象实时交互。在AR教学系统项目中,AR技术用于提供直观、生动的学习体验;在AR助手项目中,AR技术为企业员工和研发人员提供动态、实时信息。

\subsubsubsection{人工智能与机器人技术}

包括机器学习、自然语言处理、计算机视觉等多个领域的技术,以及机器人的设计、制造和控制技术。在人机协同技术实验项目中,人工智能和机器人技术用于实现人与机器之间的无缝协作;在云实习系统中,人工智能可用于智能推荐和辅导功能。

\subsubsubsection{云计算技术}
通过互联网提供计算资源和服务,包括数据存储、处理、分析等功能。在云实习系统项目中,云计算技术用于提供远程实习环境和服务;在研究院的其他项目中,云计算技术也可用于数据存储和分析等工作。

\subsection{清华大学人因工程与智能交互研究所}\label{清华大学人因工程与智能交互研究所}


\subsubsection{项目名称}

人因工程与智能交互实验平台研发

\subsubsection{项目内容}

\subsubsubsection{知识库构建与知识聚合}
研究所从人的生理、心理及社会属性出发,构建基于人因工程原理的典型工程实验知识库,通过知识图谱技术实现知识的深度整合与智能推荐。
\subsubsubsection{虚拟实验环境设计}
研发了支持虚实互联的虚拟实验操作系统,注重用户体验的优化,确保实验操作的直观性和便捷性。
\subsubsubsection{实验步骤生成与引导}
结合场景化感知和启发式推理技术,设计智能实验步骤生成算法,为学生提供个性化的实验指导。
\subsubsubsection{多传感器融合分析}
在实验中集成多种传感器,实现实验数据的实时采集与分析,提高实验过程的可监测性和可控性。
\subsubsubsection{平台研发与应用}
构建了人因工程与智能交互实验平台,并在教育、工业设计、安全评估等多个领域开展应用示范,推动相关技术的普及与推广。

\subsubsection{相关技术}

\subsubsubsection{多模态交互技术}

多模态交互意图理解:研究如何理解和解释用户通过语音、手势、面部表情等多种模态表达的交互意图,以实现更加自然和高效的交互体验。

语义融合技术:开发轻量化多模态交互语义融合嵌入式计算系统,提升人与非医用协作机器人的语义级交互效率。
\subsubsubsection{认知工程与人机交互建模}

人的建模与认知工程:包括人的认知建模、仿真和行为预测,信息感知与视觉搜索,决策、认知负荷与情境意识,分布式认知与团队元认知等,以深入理解人的认知和行为规律。

人机交互界面设计:研究以用户为中心的设计方法,信息产品的可用性与用户体验,支持特殊人群(如老年人)的信息产品设计,以及人机界面的创新设计等。
\subsubsubsection{工效学产品设计}

工效学评估与优化:应用计算机模拟与数据挖掘、认知神经科学等方法,对人的生理、心理、行为和社会属性与特点进行研究,并将结果应用于作业、产品、工业系统以及周边环境的设计与改善。

人体测量与生物力学:关注现代数字化人体测量理论与技术、基于三维人体测量的产品适配设计、生物力学等热点问题,以优化产品设计,提升用户体验。
\subsubsubsection{复杂人机协同作业技术}

人机协同决策:研究复杂人机协同作业任务中的人机协同决策机制,构建机器行为模型、用户操作行为模型、任务和流程模型,支撑复杂人机紧耦合系统高效、可靠运行。

异常操作行为预测:分析复杂人机紧耦合系统运行中的人机冲突现象和特征,构建异常操作行为模型,从系统全要素以及系统全生命周期两个维度构建人因效能评估体系。
\subsubsubsection{虚拟现实与仿真技术}

虚拟情景仿真:利用虚拟现实技术营造逼真的实验情景,便于以安全、可设计、可控制、可重复和可自动记录的方式来开展行为研究,弥补社会学研究方法的不足。

人体运动跟踪与仿真:建立基于光学跟踪和磁性跟踪技术的人机交互仿真系统,分析工人的操作动作和评价工作场所的设计,测评工人仿真组装任务中的疲劳问题等。
\subsubsubsection{安全与健康研究}

驾驶安全与仿真测试:设计适用于不同研究目的的实验系统和实验场景,如大屏幕全尺度模拟器、桌面型模拟器等,研究驾驶人员认知、态度、行为和能力,以及汽车安全装置有效性评价等。

体力负荷与疲劳研究:通过问卷、实验设计与实施、仿真建模分析等方法,研究作业人员的劳动强度水平、作业休息制度的制定与改善、疲劳程度的测定与降低疲劳等。
\subsubsubsection{跨学科合作与技术创新}

跨学科研究团队:研究团队汇集了来自不同学科的专家学者,如计算机系、心理学系、生物医学工程系等,共同推动人因工程与智能交互领域的技术创新。

产学研合作:与多家国内外学术机构和重点企业保持良好合作关系,逐步形成了一个集科研、教学、合作于一体的产学研平台。

\subsection{智能学习平台研发及应用示范——亮风台(HiScene)}\label{智能学习平台研发及应用示范——亮风台(HiScene)}

亮风台(HiScene)是一家专注于增强现实(AR)技术研发与应用的高科技企业,其产品和服务涵盖多个领域,包括智能交互教育、工业制造、商业展示、文化娱乐等。

\subsubsection{项目内容}


开发基于AR技术的互动教材、虚拟实验室、职业培训和教育游戏等。

\subsubsection{相关技术}

\subsubsubsection{虚拟现实技术}

AR引擎:亮风台自主研发的AR引擎,具备高效的图像识别与跟踪、3D建模与渲染、自然交互等功能,支持多种硬件平台和操作系统。
AR开发工具:提供一系列AR开发工具和SDK,帮助开发者快速构建AR应用。
\subsubsubsection{实时交互技术}

手势识别、语音交互、启发式推理、基于场景识别和机器学习的个性化学习路径推荐

\subsubsection{核心业务}

\subsubsubsection{增强现实(AR)技术}

AR引擎:亮风台自主研发的AR引擎,具备高效的图像识别与跟踪、3D建模与渲染、自然交互等功能,支持多种硬件平台和操作系统。

AR开发工具:提供一系列AR开发工具和SDK,帮助开发者快速构建AR应用。
\subsubsubsection{智能交互教育}

AR教材与课程:通过AR技术开发互动教材和课程,增强教学内容的可视化和互动性。

虚拟实验室:提供虚拟实验室解决方案,模拟真实实验环境,支持多种学科的实验教学。

\subsubsubsection{工业应用}

AR智能制造:利用AR技术优化制造流程,提供设备维护、故障诊断、远程协作等解决方案。

工业培训:开发基于AR的培训系统,提供安全、直观的培训环境,提升员工技能。

\subsubsubsection{商业展示}

AR营销与展示:利用AR技术为品牌和产品提供创新的展示方式,增强用户体验和互动性。

沉浸式体验:开发沉浸式体验项目,应用于展览、博物馆、文化旅游等领域。

\subsubsubsection{技术优势}

自主研发:亮风台拥有完全自主知识产权的AR引擎和核心算法,具备高效的图像识别与跟踪能力。

多平台支持:亮风台的AR技术支持多种硬件平台和操作系统,包括智能手机、平板电脑、AR眼镜等。

丰富应用场景:亮风台的AR技术应用广泛,覆盖教育、工业、商业、娱乐等多个领域。

\subsubsubsection{主要成就}

技术创新:亮风台多次获得国内外技术大奖,技术水平处于行业领先地位。

市场拓展:亮风台的产品和服务已在国内外市场广泛应用,拥有众多合作伙伴和客户。

行业影响力:亮风台在AR领域具有较高的知名度和影响力,是中国AR行业的重要企业之一。

\subsubsection{在智能交互教育方面的研究}

\subsubsubsection{AR教育平台}

亮风台开发了多个基于AR技术的教育平台,这些平台可以将虚拟信息叠加到现实世界中,提供沉浸式的学习体验。通过这些平台,学生可以通过移动设备、AR眼镜等硬件设备进行互动学习。
\subsubsubsection{互动教材}

利用AR技术开发互动教材,将静态的书本内容转化为动态的三维模型和动画,帮助学生更好地理解复杂的概念。例如,生物学教材中的细胞结构、化学教材中的分子模型等,都可以通过AR技术直观展示。
\subsubsubsection{实验模拟}

亮风台的AR技术可以模拟各种科学实验,提供虚拟实验环境,解决实验设备不足和实验条件受限的问题。例如,化学实验的反应过程、物理实验的力学演示等,都可以通过AR进行逼真模拟,学生可以在虚拟环境中进行操作和观察。
\subsubsubsection{职业培训}

在职业教育和培训方面,亮风台的AR技术可以模拟真实的工作环境和操作过程,提供安全、可重复的训练场景。例如,机械维修、医疗操作等复杂的职业技能培训,都可以通过AR技术进行仿真和练习,提高学员的实际操作能力和应变能力。
\subsubsubsection{教育游戏}

亮风台还开发了一些基于AR技术的教育游戏,通过游戏化的方式提升学生的学习兴趣和参与度。这些教育游戏将知识点融入到有趣的互动场景中,使学习变得更加生动有趣。
\subsubsubsection{教师辅助工具}

亮风台为教师提供了多种AR辅助工具,帮助教师进行课堂教学和课后辅导。例如,AR投影工具可以将复杂的三维模型投射到课堂上,供教师讲解和学生互动;AR教学助手可以根据教学内容自动生成互动教学资源,辅助教师备课和授课。
\subsubsubsection{虚拟实验室}

通过构建虚拟实验室,亮风台的AR技术可以为学生提供安全、低成本的实验环境。在虚拟实验室中,学生可以进行各种科学实验,观察实验现象,记录实验数据,从而增强实验技能和科学探究能力。
\subsubsubsection{案例示范}

亮风台在多个教育机构和项目中进行了应用示范,通过具体案例展示了AR技术在教育中的应用效果。例如,与某些学校和教育机构合作,开发并实施了基于AR技术的教学方案和实验课程,取得了良好的教学效果和学生反馈。

亮风台在智能交互教育方面的研究和应用展示了AR技术在现代教育中的巨大潜力,通过提供更加生动、互动和沉浸的学习体验,极大地提升了教学效果和学生的学习兴趣。

\subsection{魔珐科技(Mofa Tech)}\label{魔珐科技(Mofa Tech)}

\subsubsection{项目名称}

智能虚拟实验教学平台

\subsubsection{项目内容}

魔珐科技致力于开发基于VR和AR技术的虚拟实验教学平台,旨在为职业教育和基础教育提供高效的实验教学解决方案。其项目内容包括:

开发沉浸式虚拟实验室,模拟真实实验环境,使学生能够在虚拟环境中进行实验操作。
提供交互式学习资源,包括实验教程、实时反馈和指导。构建一个智能学习平台,集成虚拟实验与理论知识,提升教学效果。

\subsubsection{相关技术}

虚拟现实技术:利用VR/AR技术创建沉浸式实验环境,帮助学生在虚拟空间中进行实验操作和学习。

实时交互技术:实现手势识别、语音控制和实时反馈,增强学生与虚拟实验环境的互动性。

启发式推理:基于学生的操作行为和学习进度,动态生成个性化的实验步骤和指导,提升学习效率。

\subsection{微视酷(VSTech)}\label{微视酷(VSTech)}

\subsubsection{项目名称}

虚拟仿真实验教学系统

\subsubsection{项目内容}

研发用于职业教育和基础教育的虚拟仿真实验平台,提供丰富的实验资源和指导。微视酷专注于研发虚拟仿真实验教学系统,通过将虚拟现实技术与教育内容相结合,提供创新的实验教学方式。

构建虚拟仿真实验平台,涵盖多个学科的实验内容,特别是工程和科学领域的实验。

提供互动式教学资源,包括实验指导、操作演示和即时反馈。

开发智能教学管理系统,整合实验数据与教学评估,支持教师对学生实验过程的监督和指导。

\subsubsection{相关技术}

技术:使用3D建模和虚拟现实技术创建高度仿真的实验环境,使学生能够进行沉浸式的实验操作。

实时交互技术:集成多种传感器技术,实现高保真的物理仿真和实时数据反馈,提升实验操作的真实感和互动性。

启发式推理:利用场景化感知和启发式推理技术,根据学生的实验操作和学习行为,动态生成实验步骤和指导,提高学习效果。

